%self_confining_ringlets.tex
%
%by Joe Hahn et al, jhahn@spacescience.org, 20 November 2018.
%
%draft ApJ paper on epi_int_lite simulations of self confining ringlets.
%
%to compile:
%    pdflatex self_confining_ringlets
%    bibtex self_confining_ringlets
%    pdflatex self_confining_ringlets


\documentclass[preprint]{aastex62}

%bibliography, see
%http://ads.harvard.edu/pubs/bibtex/astronat/doc/html/astronat_3.html
\usepackage{natbib}
\usepackage{amsmath}
%\citestyle{aa}

%journal
\received{not yet}
\revised{not yet}
\accepted{not yet}
\submitjournal{Somewhere, eventually}

%short names
\shorttitle{Narrow Eccentric Ringlets}
\shortauthors{Hahn et al.}


\begin{document}


%title page
\title{N-body simulations of the Self--Confinement of 
Viscous Self--Gravitating Narrow Eccentric Planetary Ringlets}

\correspondingauthor{Joseph M. Hahn}
\email{jhahn@spacescience.org}

\author{Joseph M. Hahn}
\affiliation{Space Science Institute}

\author{Douglas P. Hamilton}
\affiliation{University of Maryland}

\author{Thomas Rimlinger}
\affiliation{University of Maryland}

\author{Lucy Luu}
\affiliation{University of Maryland}


%abstract
\begin{abstract}

N-body simulations are used
to illustrate how narrow eccentric planetary ringlets can evolve into a 
self--confining state.

\end{abstract}


%keywords
\keywords{editorials, notices --- miscellaneous --- catalogs --- surveys --- update, me}


\section{Introduction}
\label{sec:intro}

Narrow eccentric planetary ringlets have properties both interesting and
not well understood: sharp edges,
sizable eccentricity gradients, and a confinement mechanism that
opposes radial spreading due to ring viscosity.
Prevailing ringlet confinement mechanisms include: 
unseen shepherd satellites (reference), periapse pinch (ref), self gravity (ref), and
self-confinement (ref). This study uses N-body simulations to show how a viscous narrow
self--gravitating ringlet can evolve into a self-confining state.

\section{Ringlet confinement mechanisms}
\label{sec:confinement}

This section will explain the pros and cons of the various ringlet confinement mechanisms,
and will then motivate the possibility that ringlets are self confining. That possibility
is explored further via numerical simulations using the epi\_int\_lite N-body integrator.

\section{epi\_int\_lite}
\label{sec:epi_int_lite}

Epi\_int\_lite is a child of the epi\_int N-body integrator that was used to
simulate the outer edge of Saturn's B ring while it is sculpted by satellite perturbations
\citep{HS13}. The new code is very similar to its parent but differs in two significant ways:
({\it i}.) epi\_int\_lite is written in python and recoded for more efficient execution, and
({\it ii}.) epi\_int\_lite uses a more reliable drift step to handle
unperturbed motion around an oblate planet (detailed in Appendix \ref{sec:Appendix A}).

Otherwise epi\_int\_lite's treatment of ring self--gravity and viscosity are identical
to that used by the parent code; see \cite{HS13} for additional details. The epi\_int\_lite 
source code is available at https://github.com/joehahn/epi\_int\_lite, and the
code's numerical quality is assessed in Appendix \ref{sec:Appendix B}
where the output of several numerical experiments are compared against theoretical expectations.

Calculations by epi\_int\_lite use natural units with gravitation constant $G=1$, 
central primary mass $M=1$, and the ringlet's inner edge has initial radius
$r_0=1$, and so the ringlet masses $m_r$ and radii $r$ quoted below are in units of $M$ and $r_0$.
Converting code output from natural units to physical units requires choosing	
physical values for $M$ and $r_0$ and multiplying accordingly, and when this text does so
it assumes Saturn's mass $M=5.68\times10^{29}$ gm and a characteristic
ring radius $r_0=1.0\times10^{10}$ cm. Simulation time $t$ is in units of $T_{\text{orb}}/2\pi$
where $T_{\text{orb}} = 2\pi\sqrt{r_0^3/GM}$ is the orbit period at $r_0$, 
so divide simulation time $t$ by $2\pi$ and multiply
by $T_{\text{orb}}$ to convert simulation time from natural to physical units.
The simulated particles' motions during the drift step are also
sensitive to the $J_2$ portion of the primary's non-spherical gravity component 
(see Appendix \ref{sec:Appendix B}), and all simulations
adopt Saturn-like values of $J_2=0.01$ and $R_p=r_0/2$ where $R_p$ is the planet's
mean radius.

Initially all particles are assigned to various streamlines across the simulated ringlet. A streamline
is a closed eccentric path around the primary, and the $N_p$ particles in a given
streamline are initially assigned a common semimajor axis $a$ and eccentricity $e$ and are
distributed uniformly in longitude. Most of the simulations described below
employ only $N_s=2$ streamlines, so that the model output can be benchmarked against
theoretical treatments that also treat the ringlet as two gravitating rings
(e.g.\ \citealt{BGT83}). But the following also performs a few higher-resolution simulations
using $N_s=11$ streamlines, to demonstrate that the $N_s=2$ treatment is perfectly
adequate and reproduces all the relevant dynamics. All simulations use $N_p=241$ particles 
per streamline, and the total number of particles is $N_sN_p$.
Note that the assignment of particles to a given streamline is merely
for labeling purposes, as particles are still free to wander in response
to the ring’s internal forces, namely, ring gravity and viscosity. But as \cite{HS13} as well
as this work shows, the simulated ring stays coherent and highly organized throughout the 
simulation such that particles on the same streamline do not pass each other longitudinally,
nor do they cross adjacent streamlines. Because the simulated ringlet stays highly organized,
there is no radial or longitudinal mixing of the ring particles, and simulated particles preserve
their streamline membership over time.

The epi\_int\_lite N-body integrator uses the same drift-kick
scheme utilized by the MERCURY Nbody algorithm \citep{C99}, except that the
epi\_int\_lite particles that do not interact with each other directly.
Rather, the epi\_int\_lite particles
are only perturbed by the accelerations exerted by the ringlet's streamlines. 
Those accelerations are sensitive to the streamline's relative motions and
orientations, with those being inferred from the particles' positions
and velocities. Epi\_int\_lite particles are thus trace particles
that indicate where the ringlet's streamlines' reside, with the epi\_int\_lite
N-body integrator used to compute the orbital evolution of those trace particles
due to ringlet perturbations. The streamline approach used here allows one to
compute the global evolution of the ringlet using a very modest numbers of trace particles, 
typically $\sim500$.

The simulations described below account for streamline gravity
and ringlet viscosity. Because a ringlet is narrow, all particles
are in close proximity to the nearby portions of all streamlines,
which allows to approximate a streamline as an infinitely
long wire of matter having linear density $\lambda$.
Thus a particle experiences gravitational acceleration $A_g$
towards the perturbing streamline, where
\begin{equation}
\label{eqn:gravity}
    A_g=\frac{2G\lambda}{\Delta }
\end{equation}
and $\Delta$ is the particle's distance from the perturbing streamline.
A hydrodynamic approximation is used to account
for the dissipative influences exerted by adjacent particle streamlines
as they shear past the perturbed particle, and that dissipation is
parameterized by the ringlet's kinematic
shear viscosity $\nu_s$ and its kinematic bulk visocity $\nu_b$. 
The acceleration due to the ringlet's shear viscosity viscosity is
\begin{equation}
\label{eqn:shear_viscosity}
    A_{\nu_s}=-\frac{1}{\sigma r}\frac{\partial {\cal F}_L}{\partial r}
\end{equation}
where $r$ is the particle's radial coordinate and ${\cal F}_L$ is the ringlet's viscous flux of angular momentum,
\begin{equation}
    \label{eqn:F_nu_theta}
    {\cal F}_{L,\nu} = -\nu_s \sigma r^2\frac{\partial\omega}{\partial r}
\end{equation}
where $\nu_s$ is the ringlet's kinematic shear viscosity, $\sigma$ is the surface density of ringlet matter,
and $\omega$ is the perturbed particle's angular velocity,
and {\it iii.}) acceleration due to ringlet pressure $p$ is
\begin{equation}
\label{eqn:pressure}
    A_p=-\frac{1}{\sigma }\frac{\partial p}{\partial r}.
\end{equation}


\section{N-body simulations of viscous gravitating ringlets}
\label{sec:nbody}

This Section describes a suite of N-body simulations of narrow viscous gravitating planetary
ringlets, to highlight the range of initial ringlet conditions the do evolve into
a self-confining state, and those that do not.

\subsection{nominal model}
\label{subsec:nominal}

Figure \ref{fig:a_nominal} shows the semimajor axis evolution of what is referred to
as the nominal model since this ringlet readily evolves into a self-confining state.
The simulated ringlet is composed of $N_s=2$ streamlines having $N_p=241$ particles
per streamline, and the integrator timestep is $\Delta t=0.5$ in natural units, so
the integrator samples the particles' orbits $2\pi/\Delta t\simeq13$ times per orbit, and this
ringlet is evolved for $4.7\times10^3$ orbits, which requires 15 minutes execution time
on a 5 year old laptop. The ringlet's mass is
$m_r=5\times10^{-10}$, its shear viscosity is $\nu_s=2.5\times10^{-12}$, and its
bulk viscosity is $\nu_b=\nu_s$. The ringet's initial radial width is
$\Delta a_0 = 3\times10^{-4}$, its initial eccentricity is $e=0.01$, and its
eccentricity gradient is initially zero. A convenient measure of time is the ringlet's
viscous radial spreading timescale
\begin{equation}
\label{eqn:viscous-timesscale}
    \tau_\nu=\frac{\Delta a_0^2}{12\nu_s}, 
\end{equation}
which can be inferred from Eqn.\ (2.13) of \cite{P81}. 
This simulation's viscous timescale is $\tau_\nu=3.0\times10^3$ in natural units
or $\tau_\nu/2\pi=4.8\times10^2$ orbital periods. If this ringlet were orbiting Saturn
at $r_0=1.0\times10^{10}$ cm then the simulated ringlet's physical mass
would be $m_r=2.8\times10^{20}$ gm which is equivalent to the mass of a $41$ km radius iceball assuming
a volume density $\rho=1$ gm/cm$^3$, and the ringlet's initial radial width would be
$\Delta a_0 = 3\times10^{-4}r_0=30$ km. This ringlet's
orbit period would be $T_{orb}=2\pi\sqrt{r_0^3/GM}=9.0$ hours in physical units, so 
the ringlet's viscous timescale is $\tau_\nu=12$ years, and
so its shear viscosity is $\nu_s=\Delta a_0^2/12\tau_\nu = 4.8\times10^4$ cm$^2$/sec
when evaluated in physical units. 
This ringlet's initial surface density would be $\sigma=m_r/2\pi r_0\Delta a_0=1500$ gm/cm$^2$, but
Figs.\ \ref{fig:a_nominal}--\ref{fig:da_nominal} show that shrinks by a factor of 4 as the 
ringlet's sememajor axis width $\Delta a$ grows via viscous spreading until it settles into
the self-confining state at time $t\sim20\tau_\nu$.
This so-called nominal ringlet is probably overdense and overly viscous compared to known 
planetary ringlets,
but that is by design so that the simulated ringlet quickly settles into the self-confining state.
Section XX also shows how outcomes scale when a wide variety of alternate initial masses, orbits,
and viscosities are also considered.

\begin{figure}
\plotone{figs/a_nominal.pdf}
\caption{Evolution of the nominal ringlet's semimajor axes $a$
versus time $t$ in units of the ringlet's viscous time $\tau_\nu$.
This ringlet is composed of $N_s=2$ streamlines,
and the outer (blue) and inner (green) streamlines' semimajor axes are plotted relative
to their mean $a_{\text{mean}}$, and displayed in units of the ringlet's
initial width $\Delta a_0 = 3\times10^{-4}$ in natural units ({\it i.e.}\ $G=M=r_0=1$).
The simulated ringlet has total mass $m_r=5\times10^{-10}$, shear viscosity $\nu_s=2.5\times10^{-12}$,
and initial eccentricity $e=0.01$. See Section \ref{subsec:nominal} to convert
$m_r$, $a$ and $\nu_s$ from natural units to physical units.
\label{fig:a_nominal}}
\end{figure}

\begin{figure}
\plotone{figs/da_nominal.pdf}
\caption{
\label{fig:da_nominal}
The nominal ringlet's semimajor axis width $\Delta a = a_{\text{outer}} - a_{\text{inner}}$ over time
and in units of its initial radial width $\Delta a_0$.}
\end{figure}

Figure \ref{fig:e_nominal} shows that the outer streamline's eccentricity initially grows at the
expense of the inner streamline's, and this is a consequence the self-gravitating ringlet's
secular perturbations of itself, which is also demonstrated in Appendix \ref{sec:Appendix D}. 
Figure \ref{fig:de_nominal} shows
the ringlet's eccentricity difference $\Delta e = e_{\text{outer}} - e_{\text{inner}}$
and longitude of periapse difference
$\Delta\tilde{\omega} = \tilde{\omega}_{\text{outer}} - \tilde{\omega}_{\text{inner}}$,
which both settle into equilibrium values after the ringlet arrives at the self-confining
state.

\begin{figure}
\plotone{figs/e_nominal.pdf}
\caption{
\label{fig:e_nominal}
The nominal ringlet's eccentricity evolution.}
\end{figure}

\begin{figure}
    \plotone{figs/de_dwt_nominal.pdf}
    \caption{
        \label{fig:de_nominal}
        The nominal ringlet's eccentricity difference $\Delta e = e_{\text{outer}} - e_{\text{inner}}$
        and longitude of periapse difference
        $\Delta\tilde{\omega} = \tilde{\omega}_{\text{outer}} - \tilde{\omega}_{\text{inner}}$
        in radians divided by 10.
    }
\end{figure}

Figure \ref{fig:nominal_streamlines} shows the radii of the ringlet's two streamlines plotted
versus their relative longitude $\varphi=\theta-\tilde{\omega}_{\text{inner}}$ at time $t=100\tau_\nu$
when the simulation ends. In all simulations examined here, the ringlet's periapse twist 
$\Delta\tilde{\omega} = \tilde{\omega}_{\text{outer}} - \tilde{\omega}_{\text{inner}}$ is negative,
so the outer streamline's longitude of periapse $\tilde{\omega}$ trails
the inner streamline's, which in turn causes the streamlines' separations along
the ringlet's pre-periapse side (where $\varphi < 0$) to be smaller than at post-periapse ($\varphi>0$).
Which makes the ringlet's surface density asymmetric, with maximum surface density
occurring just prior to periapse, see Figs.\ \ref{fig:nominal_streamlines}--\ref{fig:radial_sigma_nominal}.

\begin{figure}
    \plotone{figs/nominal_streamlines.pdf}
    \caption{
        \label{fig:nominal_streamlines}
        The radii of the nominal ringlet's streamlines are plotted versus relative longitude
        $\varphi=\theta-\tilde{\omega}$ at time $t=100\tau_\nu$, with $\Delta a$ being the streamlines'
        semimajor axis difference then. Inset plot
        shows outer streamline's longitude of periapse $\tilde{\omega}$ trailing
        the inner streamline's.
    }
\end{figure}

\begin{figure}
    \plotone{figs/nominal_sigma_vs_longitude.pdf}
    \caption{
        \label{fig:nominal_sigma_vs_longitude}
        Nominal ringlet's surface density $\sigma(\varphi)$ is plotted versus relative
        longitude $\varphi$ at selected times $t$. Note that the ringlet's surface
        density maxima occurs just before peripase,  and is due to the ringlet's
        negative periapse twist 
        $\Delta\tilde{\omega} = \tilde{\omega}_{\text{outer}} - \tilde{\omega}_{\text{inner}} < 0$.
    }
\end{figure}

\begin{figure}
    \plotone{figs/radial_sigma_nominal.pdf}
    \caption{
        \label{fig:radial_sigma_nominal}
        Radial profiles of the nominal ringlet's surface density $\sigma(\varphi)$ at time $t/\tau_\nu=100$
        when the ringlet is self-confining. Each surface density profile is plotted versus radial distance $r$ 
        relative to $r_{mid}$, which is the ringlet's midpoint along relative longitude $\varphi = \theta-\tilde{\omega}$,
        with those radial distances $r - r_{mid}$ measured in units of the ringlet's final semimajor axis width $\Delta a$,
        and surface density is shown in units of the ringlet's longitudinally-averaged surface density $\sigma_0$.
        Radial surface density profiles are plotted along the ringlet's periapse ($\varphi=0$, blue curve), which is 
        where the ringlet's streamlines are most concentrated and surface denisity $\sigma$ is
        greatest due to the ringlet's eccentricity gradient $e'$, at the pre-periapse
        quadrature ($\varphi=-\pi/2$, red curve), post-periapse
        quadrature ($\varphi=\pi/2$, green curve)
        and at apoapse ($|\varphi|=\pi$, orange curve) where streamlines have their greatest separation
        and ringlet surface density is lowest. This ringlet's surface density contrast, between periapse and
        apoapse, is 14.
    }
\end{figure}

It is convenient to recast these orbit element differences as dimensionless gradients
\begin{equation}
    \label{eqn:e_prime}
    e' = a\frac{de}{da}
    \qquad\mbox{and}\qquad
    \tilde{\omega}' = ea\frac{d\tilde{\omega}}{da}
\end{equation}
as these are the terms that contribute to the nonlinearity parameter of \cite{BGT83}:
\begin{equation}
    \label{eqn:q}
    q = \sqrt{e'^2 + \tilde{\omega}'^2}.
\end{equation}
See also Fig.\ \ref{fig:de_prime_nominal} which plot's the nominal
ringlet's dimensionless eccentricity gradient $e'$, dimensionless periapse twist $\tilde{\omega}'$,
and nonlinearity parameter $q$ versus time. All simulations examined here have
$|\tilde{\omega}'|\ll|e'|$ so that $q\simeq|e'|$, and all simulated self-confining ringlets
have a positive eccenticity gradient and a negative periapse twist such that
the outer ringlet's periapse trails the inner ringlet's, consistent with the findings of
\cite{BGT83}.\vfil

\begin{figure}
    \plotone{figs/de_prime_nominal.pdf}
    \caption{
        \label{fig:de_prime_nominal}
        The nominal ringlet's dimensionless eccentricity gradient $e' = a\Delta e/\Delta a$
        (blue curve), dimensionless periapse twist $\tilde{\omega}' = ea\Delta\tilde{\omega}/\Delta a$
        (orange), and nonlinearity parameter $q=\sqrt{e'^2 + \tilde{\omega}'^2}$
        (green) versus time $t/\tau_\nu$. Dotted red line 
        is the threshold for self-confinement in a non-gravitating ringlet, $e'=\sqrt{3}/2\simeq0.866$
    }
\end{figure}\vfil


\section{angular momentum and energy fluxes, and luminosities}
\label{sec:fluxes}

The above evolution is readily understood when the ringlet's flux of angular momentum and energy are considered. 

\subsection{angular momentum and energy fluxes}
\label{subsec:fluxes}

The torque that is exerted on a small streamline segment of mass $\delta m$
at location $\mathbf{r}=r\hat{\mathbf{r}}$
due to the streamlines orbiting interior to it is $\delta T=\delta m\mathbf{r}\times\mathbf{A}^1$ where 
$\mathbf{A}^1=A^1_r\hat{\mathbf{r}} + A^1_\theta\hat{\boldsymbol\theta}$
is the so-called one-sided acceleration that is exerted on $\delta m$ by the interior streamlines. 
Since $\delta m=\lambda\delta\ell$ where $\lambda$ is the streamline's linear mass density,
and $\delta\ell$ is the segment's length, the ringlet's radial flux of angular momentum is then
\begin{equation}
    \label{eqn:F_L}
    {\cal F}_L(r, \theta) = \frac{\delta T}{\delta\ell} = \lambda r A^1_\theta,
\end{equation}
where $A^1_\theta$ is the tangential component of the one-sided acceleration.
A simulated ringlet total mass $m_r$ distributed across $N_s$ streamlines will have a
linear mass density $\lambda=m_r/N_s/2\pi a$.

The work that the interior streamlines exert on $\delta m$ as that segment travels a small distance
$\delta\mathbf{r}=\mathbf{v}\delta t$ in time $\delta t$
is $\delta W=\delta m\mathbf{A}^1\cdot\delta\mathbf{r}$ where 
$\mathbf{v}=v_r\hat{\mathbf{r}} + v_\theta\hat{\boldsymbol\theta}$ is the segment's velocity, and
that work accrues at $\delta m$ at the rate 
$\delta W/\delta t=\lambda\mathbf{A}^1\cdot\mathbf{v}\delta\ell$,
so the radial flux of energy through that ringlet segment is 
\begin{equation}
    \label{eqn:F_E}
    {\cal F}_E(r, \theta) = \frac{\delta W}{\delta\ell\delta t} = \lambda\mathbf{A}^1\cdot\mathbf{v},
\end{equation}
and is due to the ringlet's viscosity and self-gravity
{\it i.e.}\ ${\cal F}_E = {\cal F}_{E,\nu} + {\cal F}_{E,g}$. 

\subsection{luminosities}
\label{subsec:luminosities}

The streamline containing segment $\delta m$ has semimajor axis $a$, and 
integrating the radial angular momentum flux ${\cal F}_L$ about the entire streamline $a$
then yields the ringlet's radial angular momentum luminosity,
\begin{equation}
    \label{eqn:L_L}
    {\cal L}_L(a) = \oint {\cal F}_Ld\ell , 
\end{equation}
which is the torque that is exerted on streamline $a$ by those orbiting interior to it. Similarly,
integrating the radial energy flux ${\cal F}_E$ about streamline $a$ also yields the ringlet's radial energy luminosity
\begin{equation}
    \label{eqn:L_E}
    {\cal L}_E(a) = \oint {\cal F}_Ed\ell,
\end{equation}
and this is the rate that streamlines interior to $a$ communicate energy to the streamline just beyond.

\subsection{viscous angular momentum transport}
\label{subsec:viscous_flux}

Angular momentum is transported radially through the ring via viscosity and self-gravity, 
so ${\cal F}_L={\cal F}_{L,\nu} + {\cal F}_{L,g}$,
where the ringlet's viscous flux of angular momentum is
\begin{equation}
    \label{eqn:F_nu_theta}
    {\cal F}_{L,\nu}(r, \theta) = -\nu_s \sigma r^2\frac{\partial\omega}{\partial r}
\end{equation}
\citep{HS13} when written as a function of spatial coordinates and
angular velocity $\omega=\dot{\theta}$ (Eqn.\ XX). If we consider a small arc of 
ring material of transverse length $d\ell$, 
then ${\cal F}_{L,\nu}d\ell$ would be the torque that arc exert exerts on
ring matter just exterior, due to viscous friction,
so that is the rate that friction transmits angular momentum radially across that arc.
And when ${\cal F}_{L,\nu}$ is evaluated along a single eccentric streamline of semimajor axis $a$, 
the above simplifies to
\begin{equation}
    \label{eqn:F_nu_varphi}
    {\cal F}_{L,\nu}(a, \varphi) = {\cal F}_{L,\nu,c}\frac{1-\frac{4}{3}e'\cos\varphi}{(1-e'\cos\varphi)^2}
\end{equation}
where $\varphi=\theta-\tilde{\omega}$ is the longitude relative to periapse
and ${\cal F}_{L,\nu,c}=\frac{3}{2}\nu_s\sigma_0a\Omega$ is the viscous angular momentum flux through a
circular streamline of semimajor axis $a$ and angular speed $\Omega(a)$,
with Eqn.\ (\ref{eqn:F_nu_varphi}) assuming  that
$|\tilde{\omega}'|\ll e'$ so that $q\simeq e'$
(see \citealt{BGT82} and Appendix \ref{sec:Appendix E}). Integrating the above
around the streamline's circumference then yields its angular momentumum luminosity,
\begin{equation}
    \label{eqn:L_nu}
    {\cal L}_{L,\nu}(a) = \oint {\cal F}_{L,\nu}(a, \varphi)rd\varphi = {\cal L}_{L,\nu,c}\frac{1-\frac{4}{3}e'^2}{(1-e'^2)^{3/2}},
\end{equation}
which is the torque that one streamline exerts on its exterior neighbor due to viscous friction
(\citealt{BGT82} and Appendix \ref{sec:Appendix E}),
where ${\cal L}_{L,\nu,c}=3\pi\nu_s\sigma_0a^2\Omega$ is the viscous angular momentum luminosity of a circular streamline.

\citet{BGT82} examine angular momentum transport through a viscous eccentric but non-gravitating ringlet, 
and use Eqns.\ (\ref{eqn:F_nu_varphi}--\ref{eqn:L_nu}) to show that this transport has three 
regimes distinguished by the ringlet's $e'$:

\begin{enumerate}

\item $e'<3/4$, so the ringlet's viscous angular momentum flux ${\cal F}_{L,\nu}(\varphi)>0$
at all ringlet longitudes $\theta$. The ringlet's viscous
angular momentum luminosity ${\cal L}_{L,\nu}>0$, so viscous friction transports angular momentum radially outwards,
and the inner ring matter evolves to smaller orbits while exterior ring matter evolves outwards, and
the ringlet spreads radially.

\item $3/4\le e'<\sqrt{3}/2$. In this regime there is a range of longitudes $\theta$
where the viscous angular momentum flux is reversed such that ${\cal F}_{L,\nu}(\varphi)<0$. 
That angular momentum flux reversal is due to the $\partial\omega/\partial r$
term in Eqn.\ (\ref{eqn:F_nu_theta}) changing sign near periapse when $e'>0.75$; 
see Fig.\ \ref{fig:nominal_shear}.
Nonetheless ${\cal L}_{L,\nu}$, which is proportional to the orbit-average of ${\cal F}_{L,\nu}(\varphi)$,
is positive and the ringlet still spreads radially, albeit slower than when $e'<0.75$.

\item $e'>\sqrt{3}/2$. Viscous angular momentum flux reversal is complete such that ${\cal L}_{L,\nu}<0$,
viscous friction transports angular momentum radially inwards, and the ringlet
shrinks radially. But if $e'=\sqrt{3}/2\simeq0.866$ then ${\cal L}_{L,\nu}=0$ and the ringlet's
radial evolution ceases, and the viscous but non-gravitating ringlet is self confining.

\end{enumerate}
 
Note though that the nominal ringlet's eccentricity gradient
exceeds the $e'=\sqrt{3}/4\simeq0.866$ threshold (which is
the dotted red line in Fig.\ \ref{fig:de_prime_nominal}) when it settles into
self-confinement. This is due to the ringlet's self-gravity,
which also transports a flux of angular momentum ${\cal F}_{L,g}$ radially through the ringlet.

Figure \ref{fig:F_nu_nominal} shows the nominal ringlet's viscous angular momentum flux
${\cal F}_{L,\nu}$ versus relative longitude $\varphi=\theta-\tilde{\omega}$ at selected times $t$.
Early in the ringlet's evolution when time $t \le 8\tau_\nu$ (blue, orange, green, red,
and purple curves),
the ringlet is in regime 1 since $e'<0.75$ and ${\cal F}_{L,\nu}(\varphi)>0$ at all longitudes.
But by time $t = 10\tau_\nu$ (brown curve), this ringlet's eccentricity gradient exceeds $0.75$,
and angular momentum flux reversal ${\cal F}_{L,\nu}(\varphi)<0$ occurs near periapse where $|\varphi|\simeq0$
where the ringlet is most overdense due to its eccentricity gradient, see 
also Fig.\ \ref{fig:radial_sigma_nominal};
this ringlet is in regime 2 and its radial spreading is reduced by angular momentum flux reversal. 
And by time $t = 20\tau_\nu$ (yellow curve), this ringlet is seemingly in regime 3
since $e'=0.866$, so one might expect the ringlet's spreading to have stalled by
now, but keep in mind that the above analysis ignores any transport
of angular momentum via ringlet self--gravity. Figure \ref{fig:da_nominal}
shows that this gravitating ringlet's spreading has ceased soon after time $t \simeq35\tau_\nu$, 
at which point $e'=0.88$ (cyan curve), angular momentum flux reversal is nearly complete,
with the ringlet's total angular momentum luminosity ${\cal L}_L={\cal L}_{L,\nu}+{\cal L}_{L,g}$ is very close to zero.

Figure \ref{fig:angular_momentum_luminosity_nominal} and
Fig.\ \ref{fig:angular_momentum_luminosity_zoom_nominal} show that, when the ringlet is self-confining
at times $t\gg35\tau_\nu$, its positive viscous angular momentum luminosity ${\cal L}_{L,\nu}\simeq0.0085{\cal L}_{L,\nu,c}$ 
is nearly but not quitely counterbalanced
by its negative gravitational angular momentum luminosity ${\cal L}_{L,g}\simeq-0.0075{\cal L}_{L,\nu,c}$. 
That ${\cal L}_{L,\nu}$ and ${\cal L}_{L,g}$
are both offset slightly from zero also tells us that ringlet self-gravity causes
the streamline's shape and/or orientations differ 
slightly from the non-gravitating solution of \citet{BGT82}. Interestingly, 
Fig.\ \ref{fig:angular_momentum_luminosity_zoom_nominal} also shows that ${\cal L}_{L,\nu}+{\cal L}_{L,g}$ does not sum
precisely to zero, {\it i.e.}\ 
${\cal L}_{L}={\cal L}_{L,\nu}+{\cal L}_{L,g}\simeq0.001{\cal L}_{L,\nu,c}$, 
yet Section \ref{subsec:gravitational_flux}
will show that this ringlet's energy luminosity ${\cal L}_E$ is zero. Evidently it is 
${\cal L}_E$ that must be zero (rather than ${\cal L}_L$) in order for a viscous gravitating ringlet
to be self-confining, since ${\cal L}_E=0$ is required for the
streamlines' semimajor axes $a$ to not evolve relative to each other.
That this ringlet's ${\cal L}_L$ is slightly nonzero while ${\cal L}_E=0$ also implies that this ringlet's
eccentricities are still slowly evolving despite the 
self-confinement, which is evident in inset Fig.\ \ref{fig:de_nominal_zoom}.

\begin{figure}
    \plotone{figs/nominal_shear.pdf}
    \caption{
        \label{fig:nominal_shear}
        The nominal ringlet's angular shear $\partial\omega/\partial r$ is plotted versus relative 
        longitude $\varphi$ at selected moments in time; this quantity is negative when the inner streamline
        has the higher angular speed $\omega=v_\theta/r$. When the simulation starts, this nearly circular
        ringlet has eccentricity gradient $e'=0$, so $\partial\omega/\partial r\simeq-3\Omega/2r\simeq-1.5$ 
        when evaluated natural units (blue curve).
        The ringlet's $e'$ then grows over time (orange, green, red curves),
        which reverses the sign of $\partial\omega/\partial r$
        near periapse when $e'>0.75$; here the inner ringlet's angular speed
        is slower than the outer ringlet, and viscous friction causes angular momentum to instead flow inwards
        at these longitudes. 
    }
\end{figure}

\begin{figure}
    \plotone{figs/F_nu_nominal.pdf}
    \caption{
        \label{fig:F_nu_nominal}
        The nominal ringlet's viscous angular momentum flux ${\cal F}_{L,\nu}(\varphi)$,
        Eqn.\ (\ref{eqn:F_nu_varphi}), is plotted versus ringlet
        relative longitude $\varphi=\theta-\tilde{\omega}$ about the ringlet's inner
        streamline at selected times $t/\tau_\nu$, 
        with the ringlet's eccentricy gradient $e'$ also indicated, 
        and ${\cal F}_{L,\nu,c}$ the angular momentum flux in a circular ringlet
    }
\end{figure}

\begin{figure}
    \plotone{figs/angular_momentum_luminosity_nominal.pdf}
    \caption{
        \label{fig:angular_momentum_luminosity_nominal}
        Nominal ringlet's viscous angular momentum luminosity ${\cal L}_{L,\nu}$ (blue curve) versus time $t/\tau_\nu$
        and in units of a circular ring's viscous angular momentum luminosity ${\cal L}_{L,\nu,c}$, 
        as well as the ringlet gravitational angular momentum luminosity ${\cal L}_{L,g}$ (orange curve).
    }
\end{figure}

\begin{figure}
    \plotone{figs/angular_momentum_luminosity_zoom_nominal.pdf}
    \caption{
        \label{fig:angular_momentum_luminosity_zoom_nominal}
        Figure \ref{fig:angular_momentum_luminosity_nominal} is replotted to highlight that the ringlet's 
        viscous angular momentum luminosity ${\cal L}_{L,\nu}$ (blue curve)
        always stays positive (indicating that the viscous transport of angular momentum is radially outwards) which is 
        nearly but not entirely balanced by the ringlet's negative ({\it i.e.}\ inwards) gravitational angular momentum luminosity 
        ${\cal L}_{L,g}$ (orange) after time $t\gg35\tau_\nu$. Green curve is total angular momentum luminosity 
        ${\cal L}_{L,\nu} + {\cal L}_{L,g}\simeq0.001{\cal L}_{L,\nu,c}$.
    }
\end{figure}

\begin{figure}
    \plotone{figs/de_nominal_zoom.pdf}
    \caption{
        \label{fig:de_nominal_zoom}
        The nominal ringlet's eccentricity difference $\Delta e = e_{\text{outer}} - e_{\text{inner}}$ 
        from Fig. \ref{fig:de_nominal}, 
        with inset plot showing that $\Delta e$ continues to slowly grow even after self-confinement
        is established. 
    }
\end{figure}

\begin{figure}
    \plotone{figs/F_vs_longitude_nominal.pdf}
    \caption{
        \label{fig:F_vs_longitude_nominal}
        The nominal ringlet's viscous angular momentum flux ${\cal F}_{L,\nu}(\varphi)$ (blue curve) is computed
        via Eqn.\ (\ref{eqn:F_nu_theta}) and plotted in units of a circular ringlet's flux ${\cal F}_{L,\nu,c}$
        and versus relative longitude $\varphi$ as the simulation's end at time $t=100\tau_\nu$, 
        as well as the ringlet's gravitational angular momentum flux ${\cal F}_{L,g}(\varphi)$
        (orange curve via Eqn.\ \ref{eqn:F_L}).
    }
\end{figure}


\subsection{gravitational transport}
\label{subsec:gravitational_flux}

The ringlet's viscous ${\cal F}_{L,\nu}$ and gravitational ${\cal F}_{L,g}$
angular momentum fluxes are shown Fig.\ \ref{fig:F_vs_longitude_nominal}. That Figure
shows how viscous friction tends to transport angular momentum radially inwards, ${\cal F}_{L,\nu}(\varphi)<0$, 
at longitudes nearer periapse where $|\varphi|\sim0$, and outwards
at all other longitudes, with that flux reversal being due to the
reversal of the ringlet's angular velocity gradient (Fig.\ \ref{fig:nominal_shear}). 
Figure \ref{fig:F_vs_longitude_nominal} also shows that the ringlet's gravitational
transport of angular momentum is inwards as
ring-matter approaches periapse where $\varphi<0$, 
and is outwards ${\cal F}_{L,g}(\varphi)>0$ post-periapse, with that asymmetry being due to the ringlet's
negative periapse twist, $\tilde{\omega}'<0$ (Fig.\ \ref{fig:de_prime_nominal}).


Figure \ref{fig:nominal_energy_luminosity} shows the ringlet's viscous ${\cal L}_{E,\nu}$ 
and gravitational luminosity ${\cal L}_{E,g}$ over time.
That Figure's gravitational angular momentum luminosity is computed via
\begin{equation}
    \label{eqn:L_g}
    {\cal L}_{E,g}(a) = \oint {\cal F}_{E,g}(\varphi)rd\varphi = \oint \lambda r\mathbf{A}^1_g\cdot\mathbf{v}d\varphi
\end{equation}
where $\mathbf{A}^1_g$ is the one-sided gravitational acceleration experienced by a particle in streamline $a$.
Note that even though ${\cal F}_{E,\nu}$ and ${\cal F}_{E,g}$ have very different spatial dependances,
the influence of viscosity and gravity still conspire to sum to zero
in the orbit-integrated sense such that ${\cal L}_{E}=\oint ({\cal F}_{E,\nu}+{\cal F}_{E,g})rd\varphi=0$ 
after the ringlet has settled into the self-confining state.

Note that Fig.\ \ref{fig:nominal_energy_luminosity} also shows that 
the ringlet's gravitational energy luminosity is zero. Which is to be expected since 
the streamlines' gravitating ellipses only interact via their secular
perturbations, and secular perturbations do no work \citep{BC61}, hence ${\cal L}_{E,g}=0$.

\begin{figure}
    \plotone{figs/nominal_energy_luminosity.pdf}
    \caption{
        \label{fig:nominal_energy_luminosity}
        Nominal ringlet's viscous energy luminosity ${\cal L}_{E,\nu}$ (blue curve) versus time $t/\tau_\nu$
        and in units of a circular ring's viscous energy luminosity ${\cal L}_{E,\nu,c}$, 
        as well as the ringlet gravitational energy luminosity ${\cal L}_{E,g}$ (orange curve).
    }
\end{figure}


\acknowledgments
\section{Acknowledgments}
\label{sec:Acknowledgments}

This research was supported by the National Science Foundation via Grant No.\ AST-1313013.

\appendix

\section{Appendix A}
\label{sec:Appendix A}

Derive the more accurate drift step used by epi\_int\_lite...

\section{Appendix B}
\label{sec:Appendix B}

Compare epi\_int\_lite to theoretical predictions

\section{Appendix D}
\label{sec:Appendix D}

This examines the viscous evolution of a narrow eccentric non-gravitating
ringlet that is identical to the nominal ringlet of Section \ref{subsec:nominal} but
with ringlet self-gravity neglected and $J_2=0$.
As the orange curve in Fig.\ \ref{fig:da_nogravity} shows, the non-gravitating ringlet's
radial width $\Delta a$ grows steadily over time due to ringlet viscosity, 
long after the nominal self-gravitating ringlet (blue curve)
has settled into the self-confining state by time $t\sim15\tau_\nu$. This is due to the
ringlet's secular gravitational perturbations of itself,
which tends to excites the ringlet's outer streamline's eccentricity at the expense
of the inner streamline (see Fig.\ \ref{fig:e_nominal}) until the ringlet eccentricity gradient $e'$
(blue curve in Fig.\ \ref{fig:de_prime_nogravity}) grows beyond the
limit required for complete angular momentum flux reversal 
that results in the ringlet's radial confinement (dotted line). 
Note that viscosity also excites the non-gravitating
ringlet's eccentricity gradient some (orange curve), but not sufficiently to halt the ringlet's 
viscous spreading.

\begin{figure}
    \plotone{figs/da_nogravity.pdf}
    \caption{
        \label{fig:da_nogravity}
        Blue curve is the nominal ringlet's semimajor axis width $\Delta a$ versus time $t$,
        and this ringlet's radial spreading ceases by time $t\sim15\tau_\nu$ when it's self-gravity
        has excited the ringlet's eccentricity gradient $e'$ sufficiently; 
        see blue curve in Fig.\ \ref{fig:de_prime_nogravity}. Orange curve shows that
        the non-gravitating ringlet's $\Delta a$ grows without limit due to the ringlet's
        much lower eccentricity gradient. Note that planetary oblateness would
        cause the non-gravitating streamlines to precess differentially and eventually cross
        when $J_s>0$, so the non-gravitating simulation also sets $J_2=0$ 
        to avoid differential precession.
    }
\end{figure}

\begin{figure}
    \plotone{figs/de_prime_nogravity.pdf}
    \caption{
        \label{fig:de_prime_nogravity}
        blah
    }
\end{figure}

\section{Appendix E}
\label{sec:Appendix E}

This Appendix will use the orbit elements derived in Appendix \ref{sec:Appendix A} to
derive Eqn.\ \ref{eqn:F_nu_varphi} from \ref{eqn:F_nu_theta}, and then Eqn.\ (\ref{eqn:L_nu}).

\section{Appendix F}
\label{sec:Appendix F}

Viscous and gravitational energy transport...


\bibliography{jmh_bibliography}
\end{document}


\end{document}




\section{Trash}


\subsection{improved drift step around oblate planet}
\label{subsec:drift}


\subsection{kick}
\label{subsec:kick}



Epi\_int is a drift-kick integrator, and such integrators alternate between drifting
(ie advancing) a particle along its unperturbed trajectory, with each drift followed
by a velocity kick that accounts for all other perturbing forces such as ring self gravity
and ring viscosity. Drifting a particle efficiently along its unperturbed trajectory around
an oblate planet requires an analytic
expression for that trajectory, and epi\_int utilized the \cite{BL94} solution
that requires, at every timestep, the conversion of the particle's spatial coordinates
and velocities into geometric orbit elements, with the drifted particle's
orbit elements then converted back to spatial coordinates every timestep. That conversion is accurate
to order ${\cal O}(e^2)$ where $e$ is the particle's geometric eccentricity, but the
conversion from spatial coordinates to orbit elements and back is not reversible, which
means that the drifted particle's trajectory acquires an ${\cal O}(e^3)$ error
every timestep. Although the accumulation of this error was too slow to significantly impact
epi\_int's B ring simulations spanning $10^4$ orbit periods, this error does preclude 
using that code to simulate the much slower viscous evolution of ringlets over $10^6$ orbit periods.

To avoid this accumulation of drift errors, Section \ref{sec:drift-kick} derives an alternate set of geometric
orbit elements that describe the particle's unperturbed motion around an oblate planet. Note though
the conversion of spatial coordinates to the new geometric orbit elements is exact
and reversible, and so epi\_int\_lite's drift step is not a significant source of error. 

The chief principal guiding the developement of epi\_int\_lite is that the code be
accurate to solve the problem at hand while also being as simple
as possible so that the code can be developed, tested, and executed as swiftly as possble.
With this in mind, several simplifying approximations are made and are detailed
below in Section \ref{subsec:approximations} and they simplify code development and
shorten run times significantly. Section \ref{sec:testing} then assesses the impact
of those approximations, and shows that they are truly negligable and do not affect outcomes
or conclusions.

Although the epi\_int integrator was well-suited for evolving the
B ring edge over the $\sim3\times10^4$ orbits needed to monitor the B ring's response to Mimas
perturbations, that code lacked sufficient numerical accuracy to evolve a ringlet
during the $\sim3\times10^6$? orbits needed to track its slow radial spreading due to ringlet viscosity.
That inability was traced to epi\_int's drift step, and Section ? describes
how that is mitigated in the new code.
