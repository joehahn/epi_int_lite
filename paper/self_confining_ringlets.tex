%self_confining_ringlets.tex
%
%by Joe Hahn et al, jhahn@spacescience.org, 20 November 2018.
%
%draft ApJ paper on epi_int_lite simulations of self confining ringlets.
%
%to compile:
%    pdflatex self_confining_ringlets
%    bibtex self_confining_ringlets
%    pdflatex self_confining_ringlets


\documentclass[preprint]{aastex62}

%bibliography, see
%http://ads.harvard.edu/pubs/bibtex/astronat/doc/html/astronat_3.html
\usepackage{natbib}
\usepackage{amsmath}
%\citestyle{aa}

%journal
\received{not yet}
\revised{not yet}
\accepted{not yet}
\submitjournal{Somewhere, eventually}

%short names
\shorttitle{Narrow Eccentric Ringlets}
\shortauthors{Hahn et al.}


\begin{document}


%title page
\title{N-body simulations of the Self--Confinement of 
Viscous Self--Gravitating Narrow Eccentric Planetary Ringlets}

\correspondingauthor{Joseph M. Hahn}
\email{jhahn@spacescience.org}

\author{Joseph M. Hahn}
\affiliation{Space Science Institute}

\author{Douglas P. Hamilton}
\affiliation{University of Maryland}

\author{Thomas Rimlinger}
\affiliation{University of Maryland}

\author{Lucy Luu}
\affiliation{University of Maryland}


%abstract
\begin{abstract}

N-body simulations are used
to illustrate how a viscous self-gravitating narrow eccentric planetary ringlet can evolve into a 
self--confining state.

\end{abstract}


%keywords
\keywords{editorials, notices --- miscellaneous --- catalogs --- surveys --- update, me}


\section{Introduction}
\label{sec:intro}

Narrow eccentric planetary ringlets have properties both interesting and
not well understood: sharp edges,
sizable eccentricity gradients, and a confinement mechanism that
opposes radial spreading due to ring viscosity. To date, nearly all of the
prevailing ringlet confinement mechanisms assume that there also exists a pair of 
unseen shepherd satellites that straddle the ringlet, with those shepherds' gravities
also torquing the ringlet's edges' in a way that keeps them radially confined 
(\citealt{GT79, GT79c, GT81, CG00, ME02}), with \cite{Getal95} showing
that a single shepherd satellite can provide temporary confinement.
However the local gravitational perturbations exerted by 
the shepherds on the nearby ring material also excites prominent kinks (\citealt{Metal05}) and 
scalloping (\citealt{Wetal09}) of the ring edges,
which should invite detection of the hypothetical satellites.
That the Cassini spacecraft did not detect any shepherds near Saturn's
well-studied narrow ringlets casts doubt upon this ringlet confinement 
mechanism \citep{L18}. 

Note though that \cite{BGT82} showed that a viscous ringlet having a
a sufficiently high eccentricity gradient can in fact be self-confining,
due to a reversal of its viscous angular momentum flux, which in turn would
cause the ringlet to get narrower over time. 
That suggestion also motivates this study,
which uses the epi\_int\_lite N-body integrator to investigate whether a 
viscous and self--gravitating ringlet might evolve into a self-confining state.

\section{epi\_int\_lite}
\label{sec:epi_int_lite}

Epi\_int\_lite is a child of the epi\_int N-body integrator that was used to
simulate the outer edge of Saturn's B ring while it is sculpted by satellite perturbations
\citep{HS13}. The new code is very similar to its parent but differs in two significant ways:
({\it i}.) epi\_int\_lite is written in python and is recoded for more efficient execution, and
({\it ii}.) epi\_int\_lite uses a more reliable drift step to handle
unperturbed motion around an oblate planet (detailed in Appendix \ref{sec:Appendix A}).
Otherwise epi\_int\_lite's treatment of ring self--gravity and viscosity are identical
to that used by the parent code, and see \cite{HS13} for additional details. The epi\_int\_lite 
source code is available at https://github.com/joehahn/epi\_int\_lite, and the
code's numerical quality is benchmarked in Appendix \ref{sec:Appendix C}
where the output of several numerical experiments are compared against theoretical expectations.

Calculations by epi\_int\_lite use natural units with gravitation constant $G=1$, 
central primary mass $M=1$, and the ringlet's inner edge has initial radius
$r_0=1$, and so the ringlet masses $m_r$ and radii $r$ quoted below are in units of $M$ and $r_0$.
Converting code output from natural units to physical units requires choosing	
physical values for $M$ and $r_0$ and multiplying accordingly, and when this text does so
it assumes Saturn's mass $M=5.68\times10^{29}$ gm and a characteristic
ring radius $r_0=1.0\times10^{10}$ cm. Simulation time $t$ is in units of $T_{\text{orb}}/2\pi$
where $T_{\text{orb}} = 2\pi\sqrt{r_0^3/GM}$ is the orbit period at $r_0$, 
so divide simulation time $t$ by $2\pi$ and multiply
by $T_{\text{orb}}$ to convert simulation time from natural to physical units.
The simulated particles' motions during the drift step are also
sensitive to the $J_2$ portion of the primary's non-spherical gravity component 
(see Appendix \ref{sec:Appendix A}), and all simulations
adopt Saturn-like values of $J_2=0.01$ and $R_p=r_0/2$ where $R_p$ is the planet's
mean radius.

\subsection{streamlines}
\label{subsec:streamlines}

Initially all particles are assigned to various streamlines across the simulated ringlet. A streamline
is a closed eccentric path around the primary, and each streamline is populated by $N_p$ particles that are
initially assigned a common semimajor axis $a$ and eccentricity $e$ while
distributed uniformly in longitude. Most of the simulations described below
employ only $N_s=2$ streamlines, so that the model output can be compared against
theoretical treatments that also treat the ringlet as two gravitating streamlines
(e.g.\ \citealt{BGT83}). But the following also performs a few higher-resolution simulations
using $N_s=3-14$ streamlines, to demonstrate that the $N_s=2$ treatment is perfectly
adequate and reproduces all the relevant dynamics. All simulations use $N_p=241$ particles 
per streamline, and the total number of particles is $N_sN_p$.
Note that the assignment of particles to a given streamline is merely
for labeling purposes, as particles are still free to wander in response
to the ring’s internal forces, namely, ring gravity and viscosity. But as \cite{HS13} as well
as this work shows, the simulated ring stays coherent and highly organized throughout the 
simulation such that particles on the same streamline do not pass each other longitudinally,
nor do they cross adjacent streamlines. Because the simulated ringlet stays highly organized,
there is no radial or longitudinal mixing of the ring particles, and simulated particles preserve
memory of their streamline membership over time. 

The epi\_int\_lite code also monitors all particles and checks whether any have crossed adjacent streamlines.
If that happens the simulation is then terminated since the particles' subsequent evolution
would no longer be computed reliably.

\subsection{N-body method}
\label{subsec:N-body method}

The epi\_int\_lite N-body integrator uses the same drift-kick
scheme used by the MERCURY Nbody algorithm \citep{C99} except that
epi\_int\_lite particles that do not interact with each other directly.
Rather, epi\_int\_lite particles
are only perturbed by the accelerations exerted by the ringlet's individual streamlines. 
Those accelerations are sensitive to the streamline's relative separations and orientations, 
which are inferred from the particles' positions
and velocities. Epi\_int\_lite particles are thus tracer particles
that indicate the streamlines' locations and orientations, which the N-body
integrator uses to compute the orbital evolution of those tracer particles
due to the perturbations exerted by those streamlines. This streamline approach 
is widely used in theoretical studies of planetary rings (c.f.\ \citealt{GT79, BGT83, BGT85})
as well as in N-body studies of rings \citep{HS13, RHH16}. The great benefit of the streamline concept
in numerical work is that it allows one to swiftly track the 
global evolution of the ringlet's streamlines numerically
using only a modest numbers of trace particles, typically $N_sN_p\sim500$.

The simulations reported on here account for streamline gravity
and ringlet viscosity. Because a ringlet is narrow, all particles
are in close proximity to the nearby portions of all streamlines,
which allows us to approximate a streamline as an infinitely
long wire of matter having linear density $\lambda$.
Consequently the gravity of each perturbing streamline draws a particle
towards that streamline with acceleration
\begin{equation}
\label{eqn:gravity}
    A_g=\frac{2G\lambda}{\Delta },
\end{equation}
where $\Delta$ is the particle's distance from the streamline.

The streamline's linear density $\lambda$ is inferred from a streamline's total
mass, $m_1=m_r/N_s=\int_0^{2\pi}\lambda d\ell$
where the integration is about the streamline's circumference. Replacing
$d\ell$ with $vdt$ where $v$ is the velocity of the streamline's particles
allows replacing the spatial integral with a timewise integral,
$m_1=\int_0^{T}\lambda vdt = \lambda v T$ where the streamline's
orbital period $T=2\pi/\Omega$ and $\Omega$ its mean orbital frequency, hence
\begin{equation}
\label{eqn:lambda}
    \lambda=\frac{m_1\Omega}{2\pi v}.
\end{equation}

The hydrodynamic approximation is used here to account for
the dissipation that occurs as particles in adjacent particle streamlines
shear past and collide with the perturbed particle,
without having to monitor individual particle-particle collisions.
The particle's acceleration due to the ring particles' shear viscosity is
\begin{equation}
\label{eqn:shear_viscosity}
    A_{\nu,\parallel}=-\frac{1}{\sigma r}\frac{\partial {\cal F}_L}{\partial r},
\end{equation}
where $r$ is the particle's radial coordinate, 
$\sigma$ is the surface density of ringlet matter, and 
${\cal F}_{L,\nu}$ is the flux of angular momentum
that is transported radially across the particle's streamline
due to its collisions with particles in adjacent streamlines, {\it i.e.}\
\begin{equation}
    \label{eqn:F_nu_theta}
    {\cal F}_{L,\nu} = -\nu_s \sigma r^2\frac{\partial\omega}{\partial r}
\end{equation}
where $\nu_s$ is the ringlet's kinematic shear viscosity
and $\omega=v_\theta/r$ is the particle's angular velocity \citep{HS13}.
The acceleration $A_{\nu,\parallel }$ is parallel to the perturbed particle's streamline
{\it i.e.}\ parallel to particle's velocity vector
$\mathbf{v}=\dot{\mathbf{r}}=v_r\hat{r} + v_\theta\hat{\theta}$ 
where $\mathbf{r} = r\hat{\mathbf{r}}$ is the particle's position vector.

Dissipative collisions also transmits linear momentum in the perpendicular direction,
which results in the additional acceleration
\begin{equation}
\label{eqn:bulk_viscosity}
    A_{\nu, \perp}=-\frac{1}{\sigma}\frac{\partial {\cal G}}{\partial r},
\end{equation}
where the radial flux of linear momentum due to ringlet viscosity is
\begin{equation}
    \label{eqn:G}
    {\cal G} = -\left(\frac{4}{3}\nu_s + \nu_b\right)\sigma\frac{\partial v_r}{\partial r}
        -\left(\nu_b - \frac{2}{3}\nu_s\right)\frac{\sigma v_r}{r},
\end{equation}
$\nu_b$ is the ringlet's kinematic bulk viscosity, and $v_r$ is the paricle's radial velocity
\citep{HS13}.

In the hydrodynamic approximation there is also the acceleration due to ringlet pressure $p$
that is due to particle-particle collisions,
\begin{equation}
\label{eqn:pressure}
    A_p=-\frac{1}{\sigma }\frac{\partial p}{\partial r}.
\end{equation}
Epi\_int\_lite treats the particle ring as a dilute gas of
colliding particles for which the 1D pressure is $p=c^2\sigma$ where
c is the particles dispersion velocity. However \cite{HS13} found ring pressure
to be inconsequential in N-body simulations of Saturn's A ring, and the ringlet simulation
examined in great detail in Section \ref{subsec:nominal} 
also showed no sensitivity to pressure effects, so all other
simulations reported on here have $c=0$.

\section{N-body simulations of viscous gravitating ringlets}
\label{sec:nbody}

The folowing describes a suite of N-body simulations of narrow viscous gravitating planetary
ringlets, to highlight the range of initial ringlet conditions that do evolve into
a self-confining state, and those that do not.

\subsection{nominal model}
\label{subsec:nominal}

Figure \ref{fig:a_nominal} shows the semimajor axis evolution of what is referred to
as the nominal model since this ringlet readily evolves into a self-confining state.
The simulated ringlet is composed of $N_s=2$ streamlines having $N_p=241$ particles
per streamline, and the integrator timestep is $\Delta t=0.5$ in natural units, so
the integrator samples the particles' orbits $2\pi/\Delta t\simeq13$ times per orbit, and this
ringlet is evolved for $1.4\times10^5$ orbits, which requires 50 minutes execution time
on a ten year old laptop. The ringlet's mass is
$m_r=10^{-10}$, its shear viscosity is $\nu_s=10^{-13}$, and its
bulk viscosity is $\nu_b=\nu_s$. The ringet's initial radial width is
$\Delta a_0 = 10^{-4}$, its initial eccentricity is $e=0.01$, and its
eccentricity gradient is initially zero. A convenient measure of time is the ringlet's
viscous radial spreading timescale
\begin{equation}
\label{eqn:viscous-timesscale}
    \tau_\nu=\frac{\Delta a_0^2}{12\nu_s} 
\end{equation}
which can be inferred from Eqn.\ (2.13) of \cite{P81}. 
This simulation's viscous timescale is $\tau_\nu=8.3\times10^3$ in natural units
or $\tau_\nu/2\pi=1.3\times10^3$ orbital periods. If this ringlet were orbiting Saturn
at $r_0=1.0\times10^{10}$ cm then the simulated ringlet's physical mass
would be $m_r=5.7\times10^{19}$ gm which is equivalent to the mass of a $24$ km radius iceball assuming
a volume density $\rho=1$ gm/cm$^3$, and the ringlet's initial physical radial width would be
$\Delta a_0 = 10^{-4}r_0=10$ km. This ringlet's
orbit period would be $T_{orb}=2\pi\sqrt{r_0^3/GM}=9.0$ hours in physical units, so 
the ringlet's viscous timescale is $\tau_\nu=1.4$ years, and
so its shear viscosity is $\nu_s=\Delta a_0^2/12\tau_\nu = 1.9\times10^3$ cm$^2$/sec
when evaluated in physical units. 
This ringlet's initial surface density would be $\sigma=m_r/2\pi r_0\Delta a_0=900$ gm/cm$^2$, but
Figs.\ \ref{fig:a_nominal}--\ref{fig:da_nominal} show that shrinks by a factor of about 5 as the 
ringlet's sememajor axis width $\Delta a$ grows via viscous spreading until it settles into
the self-confining state at time $t\sim40\tau_\nu$.
This so-called nominal ringlet is probably somewhat overdense and overly viscous compared to known 
planetary ringlets,
but that is by design so that the simulated ringlet quickly settles into the self-confining state.
Section \ref{subsec:variations} also shows how outcomes vary 
when a wide variety of alternate initial masses, widths, and viscosities are also considered.

\begin{figure}
\plotone{figs/da_nominal.pdf}
\caption{
\label{fig:a_nominal}
Evolution of the nominal ringlet's semimajor axes $a$
versus time $t$ in units of the ringlet's viscous timescale
$\tau_\nu=1.3\times10^3$ orbital periods. This ringlet is composed of $N_s=2$ streamlines,
and the outer (blue) and inner (green) streamlines' semimajor axes are plotted relative
to their mean $a_{\text{mean}}$, and displayed in units of the ringlet's
initial width $\Delta a_0 = 10^{-4}$ in natural units ({\it i.e.}\ $G=M=r_0=1$).
The simulated ringlet has total mass $m_r=10^{-10}$, shear viscosity $\nu_s=10^{-13}$,
and initial eccentricity $e=0.01$. See Section \ref{subsec:nominal} to convert
$m_r$, $a$ and $\nu_s$ from natural units to physical units.}
\end{figure}

\begin{figure}
\plotone{figs/width_nominal.pdf}
\caption{
\label{fig:da_nominal}
The nominal ringlet's semimajor axis width $\Delta a = a_{\text{outer}} - a_{\text{inner}}$ over time
and in units of its initial radial width $\Delta a_0$.}
\end{figure}

Figure \ref{fig:e_nominal} shows that the outer streamline's eccentricity initially grows at the
expense of the inner streamline's, and that is a consequence the self-gravitating ringlet's
secular perturbations of itself, which is also demonstrated in Appendix \ref{sec:Appendix D}. 
Figure \ref{fig:de_nominal} shows
the ringlet's eccentricity difference $\Delta e = e_{\text{outer}} - e_{\text{inner}}$
and longitude of periapse difference
$\Delta\tilde{\omega} = \tilde{\omega}_{\text{outer}} - \tilde{\omega}_{\text{inner}}$,
which both settle into equilibrium values after the ringlet arrives at the self-confining
state. 

In all self-confing ringlet simulations examined here, the ringlet's periapse twist 
$\Delta\tilde{\omega} = \tilde{\omega}_{\text{outer}} - \tilde{\omega}_{\text{inner}}$ is negative,
so the outer streamline's longitude of periapse $\tilde{\omega}$ trails
the inner streamline's, which in turn causes the streamlines' separations along
the ringlet's pre-periapse side (where $\varphi = \theta - \tilde{\omega} < 0$) 
to differ slightly from the post-periapse ($\varphi>0$) side.
Which in turn makes the ringlet's surface density asymmetric {\it i.e.}\ is
maximal just prior to periapse, see
Figs.\ \ref{fig:nominal_sigma_vs_longitude}--\ref{fig:radial_sigma_nominal}.

\begin{figure}
\plotone{figs/e_nominal.pdf}
\caption{
\label{fig:e_nominal}
The nominal ringlet's eccentricity evolution.}
\end{figure}

\begin{figure}
    \plotone{figs/de_dwt_nominal.pdf}
    \caption{
        \label{fig:de_nominal}
        The nominal ringlet's eccentricity difference $\Delta e = e_{\text{outer}} - e_{\text{inner}}$
        and longitude of periapse difference
        $\Delta\tilde{\omega} = \tilde{\omega}_{\text{outer}} - \tilde{\omega}_{\text{inner}}$
        in radians divided by 10.
    }
\end{figure}

\begin{figure}
    \plotone{figs/nominal_sigma_vs_longitude.pdf}
    \caption{
        \label{fig:nominal_sigma_vs_longitude}
        Nominal ringlet's surface density $\sigma(\varphi)$ versus relative
        longitude $\varphi$ at selected times $t$ and plotted in units of ringlet's initial
        mean surface density $\sigma_0$. Note that the ringlet's surface
        density maxima occurs just before peripase,  and is due to the ringlet's
        negative periapse twist 
        $\Delta\tilde{\omega} = \tilde{\omega}_{\text{outer}} - \tilde{\omega}_{\text{inner}} < 0$.
    }
\end{figure}

\begin{figure}
    \plotone{figs/radial_sigma_nominal.pdf}
    \caption{
        \label{fig:radial_sigma_nominal}
        Radial profiles of the nominal ringlet's surface density $\sigma(\varphi)$ at time $t/\tau_\nu=100$
        when the ringlet is self-confining. Each surface density profile is plotted versus radial distance $r$ 
        relative to $r_{mid}$, which is the ringlet's midpoint along relative longitude $\varphi = \theta-\tilde{\omega}$,
        with those radial distances $r - r_{mid}$ measured in units of the ringlet's final semimajor axis width $\Delta a$,
        and surface density is shown in units of the ringlet's longitudinally-averaged surface density $\sigma_0$.
        Radial surface density profiles are plotted along the ringlet's periapse ($\varphi=0$, blue curve), which is 
        where the ringlet's streamlines are most concentrated and surface denisity $\sigma$ is
        greatest due to the ringlet's eccentricity gradient $e'$, at the pre-periapse
        quadrature ($\varphi=-\pi/2$, red curve), post-periapse
        quadrature ($\varphi=\pi/2$, green curve)
        and at apoapse ($|\varphi|=\pi$, orange curve) where streamlines have their greatest separation
        and ringlet surface density is lowest. This ringlet's surface density contrast between periapse and
        apoapse is about 14.
    }
\end{figure}

It is convenient to recast these orbit element differences as dimensionless gradients
\begin{equation}
    \label{eqn:e_prime}
    e' = a\frac{de}{da}
    \qquad\mbox{and}\qquad
    \tilde{\omega}' = ea\frac{d\tilde{\omega}}{da}
\end{equation}
as these are the terms that contribute to the nonlinearity parameter of \cite{BGT83}:
\begin{equation}
    \label{eqn:q}
    q = \sqrt{e'^2 + \tilde{\omega}'^2}.
\end{equation}
See also Fig.\ \ref{fig:de_prime_nominal} which plot's the nominal
ringlet's dimensionless eccentricity gradient $e'$, dimensionless periapse twist $\tilde{\omega}'$,
and nonlinearity parameter $q$ versus time. All simulated self-confining ringlets
have a positive eccenticity gradient and a negative periapse twist such that
the outer ringlet's periapse trails the inner ringlet's, consistent with the findings of
\cite{BGT83}.\vfil

\begin{figure}
    \plotone{figs/de_prime_nominal.pdf}
    \caption{
        \label{fig:de_prime_nominal}
        The nominal ringlet's dimensionless eccentricity gradient $e' = a\Delta e/\Delta a$
        (blue curve), dimensionless periapse twist $\tilde{\omega}' = ea\Delta\tilde{\omega}/\Delta a$
        (orange), and nonlinearity parameter $q=\sqrt{e'^2 + \tilde{\omega}'^2}$
        (green curve which overlaps blue) versus time $t/\tau_\nu$. Dotted red line 
        is the threshold for self-confinement in a non-gravitating ringlet, $e'=\sqrt{3}/2\simeq0.866$
    }
\end{figure}\vfil


\section{angular momentum and energy fluxes \& luminosities}
\label{sec:fluxes}

The nominal ringlet's evolution is readily understood when the ringlet's 
radial flux of angular momentum and energy are considered. 

\subsection{angular momentum and energy fluxes}
\label{subsec:fluxes}

The torque that is exerted on a small streamline segment of mass $\delta m$
at location $\mathbf{r}=r\hat{\mathbf{r}}$
due to the streamlines orbiting interior to it is $\delta T=\delta m\mathbf{r}\times\mathbf{A}^1$ where 
$\mathbf{A}^1=A^1_r\hat{\mathbf{r}} + A^1_\theta\hat{\boldsymbol\theta}$
is the so-called one-sided acceleration that is exerted on $\delta m$ by all other streamlines
interior to it. Since $\delta m=\lambda\delta\ell$ where $\lambda$ is the streamline's linear mass density,
and $\delta\ell$ is the segment's length, then the radial flux of angular momentum flowing into
that segment due to the accelerations that are exerted by streamlines interior to that
segment is
\begin{equation}
    \label{eqn:F_L}
    {\cal F}_L(r, \theta) = \frac{\delta T}{\delta\ell} = \lambda r A^1_\theta,
\end{equation}
where $A^1_\theta$ is the tangential component of the one-sided acceleration,
and the streamline's linear mass density $\lambda$ is computed using Eqn.\ (\ref{eqn:lambda}).
The radial angular momentum flux, Eqn.\ (\ref{eqn:F_L}), is due to the ringlet's viscosity and self-gravity,
so ${\cal F}_L = {\cal F}_{L,\nu} + {\cal F}_{L,g}$. 

The work that the interior streamlines exert on $\delta m$ as that segment travels a small distance
$\delta\mathbf{r}=\mathbf{v}\delta t$ in time $\delta t$
is $\delta W=\delta m\mathbf{A}^1\cdot\delta\mathbf{r}$ where 
$\mathbf{v}=v_r\hat{\mathbf{r}} + v_\theta\hat{\boldsymbol\theta}$ is the segment's velocity, and
that work accrues at $\delta m$ at the rate 
$\delta W/\delta t=\lambda\mathbf{A}^1\cdot\mathbf{v}\delta\ell$,
so the radial flux of energy entering that ringlet segment due to accelerations
exerted by the interior streamlines is 
\begin{equation}
    \label{eqn:F_E}
    {\cal F}_E(r, \theta) = \frac{\delta W}{\delta\ell\delta t} = \lambda\mathbf{A}^1\cdot\mathbf{v}.
\end{equation}
The radial energy flux is due to the ringlet's viscosity and self-gravity, so
${\cal F}_E = {\cal F}_{E,\nu} + {\cal F}_{E,g}$. 

\subsection{luminosities}
\label{subsec:luminosities}

The streamline containing segment $\delta m$ has semimajor axis $a$, and 
integrating the radial angular momentum flux ${\cal F}_L$ about the entire streamline
then yields the radial luminosity of angular momentum entering streamline $a$,
\begin{equation}
    \label{eqn:L_L}
    {\cal L}_L(a) = \oint {\cal F}_Ld\ell , 
\end{equation}
which is the torque that is exerted on streamline $a$ by those orbiting interior to it. Similarly,
integrating the radial energy flux ${\cal F}_E$ about streamline $a$ also yields the ringlet's radial energy luminosity
\begin{equation}
    \label{eqn:L_E}
    {\cal L}_E(a) = \oint {\cal F}_Ed\ell,
\end{equation}
which is the rate that the interior streamlines communicate energy to streamline $a$.

\subsection{viscous transport of angular momentum}
\label{subsec:viscous_flux}

Angular momentum is transported radially through the ring via viscosity and self-gravity, 
so ${\cal F}_L={\cal F}_{L,\nu} + {\cal F}_{L,g}$,
where the ringlet's viscous flux of angular momentum is
\begin{equation}
    \label{eqn:F_nu_1}
    {\cal F}_{L,\nu}(r, \theta) = -\nu_s \sigma r^2\frac{\partial\omega}{\partial r}
\end{equation}
when Eqn.\ (\ref{eqn:F_nu_theta}) is written as a function of spatial coordinates and
angular velocity $\omega=\dot{\theta}$. 
If we consider a small arc of  ring material of length $d\ell$, 
then ${\cal F}_{L,\nu}d\ell$ is the torque that arc exert exerts on
ring matter just exterior, due to viscous friction,
so that is the rate that friction transmits angular momentum radially across that arc.
And when ${\cal F}_{L,\nu}$ is evaluated along a single eccentric streamline of semimajor axis $a$, 
the above simplifies to
\begin{equation}
    \label{eqn:F_nu_varphi}
    {\cal F}_{L,\nu}(a, \varphi) = {\cal F}_{L,\nu,c}\frac{1-\frac{4}{3}e'\cos\varphi}{(1-e'\cos\varphi)^2}
\end{equation}
where the angular shear $\omega'=\partial\omega/\partial r$ in Eqn.\ (\ref{eqn:F_nu_1})
is derived in Appendix \ref{sec:Appendix B},
$\varphi=\theta-\tilde{\omega}$ is the longitude relative to periapse,
${\cal F}_{L,\nu,c}=\frac{3}{2}\nu_s\sigma_0a\Omega$ is the viscous angular momentum flux through a
circular streamline of semimajor axis $a$ and angular speed $\Omega(a)$,
and Eqn.\ (\ref{eqn:F_nu_varphi}) assumes that $|\tilde{\omega}'|\ll e'$ so that $q\simeq e'$
(\citealt{BGT82}). Integrating the above
around the streamline's circumference then yields its angular momentumum luminosity,
\begin{equation}
    \label{eqn:L_nu}
    {\cal L}_{L,\nu}(a) = \oint {\cal F}_{L,\nu}(a, \varphi)rd\varphi = {\cal L}_{L,\nu,c}\frac{1-\frac{4}{3}e'^2}{(1-e'^2)^{3/2}},
\end{equation}
which is the torque that one streamline exerts on its exterior neighbor due to viscous friction
(\citealt{BGT82}) with ${\cal L}_{L,\nu,c}=3\pi\nu_s\sigma_0a^2\Omega$
being the viscous angular momentum luminosity of a circular streamline.

\citet{BGT82} examine angular momentum transport through a viscous eccentric but non-gravitating ringlet, 
and use Eqns.\ (\ref{eqn:F_nu_varphi}--\ref{eqn:L_nu}) to show that this transport has three 
regimes distinguished by the ringlet's $e'$:

\begin{enumerate}

\item $e'<3/4$. The ringlet's viscous angular momentum flux ${\cal F}_{L,\nu}(\varphi)>0$
at all longitudes $\varphi$. The ringlet's viscous
angular momentum luminosity ${\cal L}_{L,\nu}>0$, so viscous friction transports angular momentum radially outwards,
and the inner ring matter evolves to smaller orbits while exterior ring matter evolves outwards, and
the ringlet spreads radially.

\item $3/4\le e'<\sqrt{3}/2$. In this regime there is a range of longitudes $\varphi$
where the viscous angular momentum flux is reversed such that ${\cal F}_{L,\nu}(\varphi)<0$. 
That angular momentum flux reversal is due to the $\partial\omega/\partial r$
term in Eqn.\ (\ref{eqn:F_nu_theta}) changing sign near periapse when $e'>0.75$, 
see Fig.\ \ref{fig:nominal_shear}.
Nonetheless ${\cal L}_{L,\nu}$, which is proportional to the orbit-average of ${\cal F}_{L,\nu}(\varphi)$,
is positive and the ringlet still spreads radially, albeit slower than when $e'<0.75$.

\item $e'\ge\sqrt{3}/2$. Viscous angular momentum flux reversal is complete such that ${\cal L}_{L,\nu}\le0$,
viscous friction transports angular momentum radially inwards, and the ringlet
shrinks radially. But if $e'=\sqrt{3}/2\simeq0.866$ then ${\cal L}_{L,\nu}=0$ and the ringlet's
radial evolution ceases, and the viscous but non-gravitating ringlet is self--confining.

\end{enumerate}
 
Note though that the nominal ringlet's eccentricity gradient
exceeds the $e'=\sqrt{3}/2\simeq0.866$ threshold 
(dotted red line in Fig.\ \ref{fig:de_prime_nominal}) when it settles into
self-confinement. This is due to the ringlet's self-gravity,
which also transports a flux of angular momentum ${\cal F}_{L,g}$ radially through the ringlet.

Figure \ref{fig:F_nu_nominal} shows the nominal ringlet's viscous angular momentum flux
${\cal F}_{L,\nu}$ versus relative longitude $\varphi=\theta-\tilde{\omega}$ at selected times $t$.
Early in the ringlet's evolution when time $t \le 10\tau_\nu$ (blue, orange, green, red, purple 
and brown curves),
the ringlet is in regime 1 since $e'<0.75$ and ${\cal F}_{L,\nu}(\varphi)>0$ at all longitudes.
But by time $t \ge 20\tau_\nu$ (pink curve), this ringlet's eccentricity gradient exceeds $0.75$,
and angular momentum flux reversal ${\cal F}_{L,\nu}(\varphi)<0$ occurs near periapse where $|\varphi|\simeq0$
where the ringlet is most overdense due to its eccentricity gradient, see 
also Fig.\ \ref{fig:radial_sigma_nominal}.
This ringlet is now in regime 2 and its radial spreading is reduced by angular momentum flux reversal. 
And by time $t = 60\tau_\nu$ (yellow curve), this ringlet is seemingly in regime 3
since $e'>0.866$, so one might expect the ringlet to start contracting
now, but keep in mind that the above analysis ignores any transport
of angular momentum via ringlet self--gravity. Figure \ref{fig:da_nominal} in fact
shows that this gravitating ringlet's spreading had ceased by time $t \simeq80\tau_\nu$, 
at which point $e'=0.88$ (Fig.\ \ref{fig:F_nu_nominal} yellow curve), angular momentum flux reversal is nearly complete,
and the ringlet's total angular momentum luminosity ${\cal L}_L={\cal L}_{L,\nu}+{\cal L}_{L,g}$ is very close to zero.
Figures \ref{fig:angular_momentum_luminosity_nominal} and \ref{fig:angular_momentum_luminosity_zoom_nominal} also show that, 
when the ringlet is self-confining at times $t\ge 80\tau_\nu$, 
its small but positive viscous angular momentum luminosity ${\cal L}_{L,\nu}\simeq0.006{\cal L}_{L,\nu,c}$ 
is counterbalanced by its negative gravitational angular momentum luminosity ${\cal L}_{L,g}\simeq -0.006{\cal L}_{L,\nu,c}$,
so radial spreading has ceased and the ringlet is self-confining. 

\begin{figure}
    \plotone{figs/nominal_shear.pdf}
    \caption{
        \label{fig:nominal_shear}
        The nominal ringlet's angular shear $\partial\omega/\partial r$ is plotted versus relative 
        longitude $\varphi$ at selected moments in time; this quantity is negative where the inner streamline
        has the higher angular speed $\omega=v_\theta/r$. When the simulation starts, the
        ringlet has eccentricity $e=0.01$ and eccentricity gradient $e'=0$ 
        so $\partial\omega/\partial r\simeq-3\Omega/2r\simeq-1.5$ when evaluated natural units (blue curve).
        The ringlet's $e'$ then grows over time (orange, green, red curves),
        which reverses the sign of $\partial\omega/\partial r$
        near periapse when $e'>0.75$; here the inner ringlet's angular speed
        is slower than the outer ringlet, and viscous friction causes angular momentum to instead flow inwards
        at these longitudes. Dashed curve is Eqn.\ (\ref{eqn:domega-dr}) with $e'=\sqrt{3}/2$
        and assuming $|\tilde{\omega}'|\ll1$.
    }
\end{figure}

\begin{figure}
    \plotone{figs/F_nu_nominal.pdf}
    \caption{
        \label{fig:F_nu_nominal}
        The nominal ringlet's viscous angular momentum flux ${\cal F}_{L,\nu}(\varphi)$,
        Eqn.\ (\ref{eqn:F_nu_varphi}), is plotted 
        in units of ${\cal F}_{L,\nu,c}$ (the angular momentum flux in a circular ringlet)
        and versus the ringlet's
        relative longitude $\varphi=\theta-\tilde{\omega}$ at selected times $t/\tau_\nu$, 
        with the ringlet's eccentricy gradient $e'$ also indicated.
    }
\end{figure}

\begin{figure}
    \plotone{figs/angular_momentum_luminosity_nominal.pdf}
    \caption{
        \label{fig:angular_momentum_luminosity_nominal}
        Nominal ringlet's viscous angular momentum luminosity ${\cal L}_{L,\nu}$ (blue curve) versus time $t/\tau_\nu$
        and in units of a circular ring's viscous angular momentum luminosity ${\cal L}_{L,\nu,c}$, 
        as well as the ringlet gravitational angular momentum luminosity ${\cal L}_{L,g}$ (orange curve).
    }
\end{figure}

\begin{figure}
    \plotone{figs/angular_momentum_luminosity_zoom_nominal.pdf}
    \caption{
        \label{fig:angular_momentum_luminosity_zoom_nominal}
        Figure \ref{fig:angular_momentum_luminosity_nominal} is replotted to show that the ringlet's 
        viscous angular momentum luminosity ${\cal L}_{L,\nu}$ (blue curve)
        always stays positive (indicating that the viscous transport of angular momentum is radially outwards)
        and is eventually balanced by the ringlet's negative ({\it i.e.}\ inwards) gravitational angular momentum luminosity 
        ${\cal L}_{L,g}$ (orange) after time $t\ge 80\tau_\nu$. Green curve is total angular momentum luminosity 
        ${\cal L}_{L,\nu} + {\cal L}_{L,g}$ whose time-average is zero when $t\ge 80\tau_\nu$.
        Dashed lines are $\pm0.006{\cal L}_{L,\nu,c}$.
    }
\end{figure}

\begin{figure}
    \plotone{figs/F_vs_longitude_nominal.pdf}
    \caption{
        \label{fig:F_vs_longitude_nominal}
        The nominal ringlet's viscous angular momentum flux ${\cal F}_{L,\nu}(\varphi)$ (blue curve) is computed
        via Eqn.\ (\ref{eqn:F_nu_theta}) and plotted in units of a circular ringlet's flux ${\cal F}_{L,\nu,c}$
        versus relative longitude $\varphi$ at the simulation's end time $t=100\tau_\nu$, 
        as well as the ringlet's gravitational angular momentum flux ${\cal F}_{L,g}(\varphi)$
        (orange curve via Eqn.\ \ref{eqn:F_L}).
    }
\end{figure}


\subsection{gravitational transport}
\label{subsec:gravitational_flux}

The nominal ringlet's viscous ${\cal F}_{L,\nu}$ and gravitational ${\cal F}_{L,g}$
angular momentum fluxes are shown Fig.\ \ref{fig:F_vs_longitude_nominal} after it has settled into the self-confining state. 
That figure shows how viscous friction tends to transport angular momentum radially inwards, ${\cal F}_{L,\nu}(\varphi)<0$, 
at longitudes nearer periapse where $|\varphi|\sim0$, and outwards
at all other longitudes, with that flux reversal being due to the
reversal of the ringlet's angular velocity gradient, Fig.\ \ref{fig:nominal_shear}. 
Figure \ref{fig:F_vs_longitude_nominal} also shows that the ringlet's gravitational
transport of angular momentum is inwards as
ring-matter approaches periapse where $\varphi<0$, 
and is outwards ${\cal F}_{L,g}(\varphi)>0$ post-periapse, with that asymmetry being due to the ringlet's
negative periapse twist, $\tilde{\omega}'<0$ (Fig.\ \ref{fig:de_prime_nominal}).

Figure \ref{fig:nominal_energy_flux} shows the ringlet's energy fluxes 
due to viscosity (blue curve) and gravity (orange) at simulation end.
Integrating these fluxes about a streamline's circumference at various times $t$ then yields the
the ringlet's viscous ${\cal L}_{E,\nu}$ and gravitational energy luminosity ${\cal L}_{E,g}$ over time,
Fig.\ \ref{fig:nominal_energy_luminosity}, where the gravitational energy luminosity is computed via
\begin{equation}
    \label{eqn:L_g}
    {\cal L}_{E,g}(a) = \oint {\cal F}_{E,g}(\varphi)rd\varphi = \oint \lambda r\mathbf{A}^1_g\cdot\mathbf{v}d\varphi
\end{equation}
where $\mathbf{A}^1_g$ is the one-sided gravitational acceleration experienced by a particle in streamline $a$.
Note that even though ${\cal F}_{E,\nu}$ and ${\cal F}_{E,g}$ have very different spatial dependances 
(see Fig.\ \ref{fig:nominal_energy_flux}),
the influence of the ringlet's viscosity and gravity still conspire such that their
orbit-integrated luminosities ${\cal L}_{E}=\oint ({\cal F}_{E,\nu}+{\cal F}_{E,g})rd\varphi$ are zero 
once the ringlet has settled into the self-confining state.

Note that Fig.\ \ref{fig:nominal_energy_luminosity} also shows that 
the ringlet's gravitational energy luminosity is zero. Which is to be expected since 
the streamlines' gravitating ellipses only interact via their secular
perturbations, and secular perturbations do no work \citep{BC61}, hence ${\cal L}_{E,g}=0$.
That this quantity evaluates to zero within $\pm5\times10^{-24}$ (in nautural units) can also be regarded
as another test of the epi\_int\_lite integrator's numerical quality.


\begin{figure}
    \plotone{figs/F_energy_vs_longitude_nominal.pdf}
    \caption{
        \label{fig:nominal_energy_flux}
        Blue curve is the nominal ringlet's viscous energy flux ${\cal F}_{E,\nu}(\varphi)$, plotted 
        in units of a circular ringlet's viscous energy flux ${\cal F}_{E,\nu,c}$ 
        and versus the ringlet's relative longitude $\varphi$ at the simulation's end time $t=100\tau_\nu$, 
        as well as the ringlet's gravitational energy flux ${\cal F}_{E,g}(\varphi)$.
    }
\end{figure}

\begin{figure}
    \plotone{figs/nominal_energy_luminosity.pdf}
    \caption{
        \label{fig:nominal_energy_luminosity}
        Nominal ringlet's viscous energy luminosity ${\cal L}_{E,\nu}$ (blue curve) versus time $t/\tau_\nu$
        and in units of a circular ring's viscous energy luminosity ${\cal L}_{E,\nu,c}$, 
        as well as the ringlet gravitational energy luminosity ${\cal L}_{E,g}$ (orange curve).
    }
\end{figure}


\subsection{variations with ringlet width, mass, and viscosity}
\label{subsec:variations}

To assess whether the nominal ringlet's evolution is typical of other ringlets
having alternate values of initial width $\Delta a$, total mass $m_r$, and shear viscosity $\nu_s$,
a survey of 571 additional ringlet simulations are executed. 
These survey ringlets are similar to the nominal ringlet 
with $N_s=2$ streamlines having $N_p=241$ particles per streamline, initial 
eccentricity $e=0.01$, initial eccentricity gradient $e'=0$, and viscosities $\nu_b=\nu_s$. 
But the survey ringlets instead have
total masses that are geometrically distributed between $10^{-14}\le m_r\le10^{-9}$,
shear viscosities distributed between $10^{-15}\le \nu_s\le 10^{-11}$,
and initial radial widths distributed between  
$2.5\times10^{-5}\le \Delta a\le2.0\times10^{-4}$. Survey results are summarized in Fig.\ \ref{fig:sim_grid_da}
where circles indicate those ringlets that do evolve into a self-confining state.
All simulations of self-confining ringlets are evolved in time until 
$10\tau_{dyn}$ where the so-called dynamical time $\tau_{dyn}$ is the moment when
the ringlet's nonlinearity parameter $q$ first exceeds $0.6$, since Fig.\ \ref{fig:de_prime_nominal}
suggests that $10\tau_{dyn}$ is sufficient time to assess whether the ringlet
has truly arrived at the self-confining state. 

\begin{figure}
    \gridline{\fig{figs/sim_grid_da_0p000025.pdf}{0.5\textwidth}{}\fig{figs/sim_grid_da_0p00005.pdf}{0.5\textwidth}{}}
    \gridline{\fig{figs/sim_grid_da_0p0001.pdf}{0.5\textwidth}{}\fig{figs/sim_grid_da_0p0001.pdf}{0.5\textwidth}{}}
    \gridline{\fig{figs/sim_grid_da_0p0002.pdf}{0.7\textwidth}{}}
    \caption{
        \label{fig:sim_grid_da}
        Outcomes for 572 ringlet simulations having a variety of ringlet
        masses $m_r$ and shear viscosities $\nu_s$, with each panel showing results
        for those ringlets having the same initial radial width
        $\Delta a=2.5\times10^{-5}$, $5\times10^{-5}$, $1\times10^{-4}$, or $2\times10^{-4}$. Circles 
        indicate those ringlets that evolve into the self-confining state, while an $\times$ is
        for those simulations that terminate early when an epi\_int\_lite particle crossed an adjacent streamline,
        and N indicates the nominal ringlet simulation. The color of each self-confining ringlet shows
        whether the ringlet settled into a high--eccentricity gradient state with $e'> 0.8$ (blue),
        or smaller eccentricity gradients $0.13> e'>0.8$ (cyan), $0.037> e'>0.13$ (green), or $e<0.037$ (yellow).
        All simulations are evolved for the greater of ten dynamical timescales $\tau_{dyn}$
        or ten viscous timescales $10\tau_{\nu}$.
        The middle-right plot is redundant and is here to trick latex into rendering all plots on same page, please ignore.
    }
\end{figure}

The $\times$ simulations in Fig.\ \ref{fig:sim_grid_da}
terminated early when an epi\_int\_lite particle crossed a 
neighboring streamline. In reality, strong 
pressure forces would have developed as adjacent streamlines converged
and enhanced particle densities and particle collisions, with ring particles
possibly rebounding off this high-density region and/or splashing vertically,
none of which is accounted for with this version of epi\_int\_lite. 
So this survey simply terminates all such simulations and flags that occurrence with an 
$\times$ in Fig.\ \ref{fig:sim_grid_da}. Keep in mind though that this does not mean
that those particular ringlets would not have evolved into a self-confining state. Instead,
the streamlines in these ringlets have evolved so close to each other that a more sophisticated
and possibly nonlinear treatment of pressure effects would have been needed in order  
to accurately assess their fates.

Each circle in Fig.\ \ref{fig:sim_grid_da} represents a self-confining ringlet whose
nonlinearity parameter settles into a value that is close to the anticipated value, $q\simeq0.9$.
But these self-confining ringlets' final eccentricity gradients $e'$ also settle 
into a spectrum of values, $0\le e' \le 0.9$, with
outcomes indicated by circle color in Fig.\ \ref{fig:sim_grid_da}:
blue for a high--eccentricity gradient ringlets having 
$e'> 0.8$, cyan for smaller eccentricity gradients $0.13>e'>0.8$, green for $0.037>e'>0.13$,
and yellow for very low eccentricity gradients $e<0.037$,
with these intervals chosen so that there are approximately 60 simulations in each color-bin.
Inspection of Fig.\ \ref{fig:sim_grid_da} also shows that ringlets having lower $e'$ have
lower masses and higher viscosities than the higher $e'$ ringlets.

Figure \ref{fig:e_prime_q_vs_time} shows how eccentricity gradient varies over time $t/\tau_{dyn}$
for a sample of  self-confining ringlet simulations. That Figure also shows 
that the range of simulated $e'$ outcomes also agrees with the 
range of eccentricity gradients observed among several of Saturn's more well-studied narrow eccentric
ringlets, namely the Maxwell, Titan, and Laplace ringlets, whose $e'$ are
indicated by the black horizontal lines.
Figure \ref{fig:e_prime_q_vs_time} also shows there is a pileup of simulated 
outcomes near Laplace's eccentricity gradient $e'\simeq0.04$, as well as a dearth of simulations
at significantly lower values of $e'$. Which indicates that the Huygens ringlet, whose $e'\lesssim0.01$, 
can not be accounted for by the self-confining mechanism considered here.

Also keep in mind that Fig.\ \ref{fig:e_prime_q_vs_time} is not an apples-to-apples comparison 
of simulated ringlets to observed ringlets, since no attempt is made here to match
the observed ringlet's semimajor-axis width $\Delta a$ to the simulated ringlet's final $\Delta a$.
Rather,  Fig.\ \ref{fig:e_prime_q_vs_time}'s main point is that many of the observed narrow
eccentric ringlets that have $e'\gtrsim0.04$ are consistent with
this model of a viscous self-gravitating ringlet that is also self-confining.
And lastly, recall that $q=\sqrt{e'^2 + \tilde{\omega}'^2}$ which tells us that lower eccentricity gradient
ringlets have a larger periapse twist $|\tilde{\omega}'|$.

\begin{figure}
    \plotone{figs/e_prime_selected.pdf}
    \caption{
        \label{fig:e_prime_q_vs_time}
        Eccentricity gradient $e'$ versus time $t/\tau_{dyn}$ for a sample of 28 self-confining
        ringlets such that their eccentricity gradients $e'$ span the range of values
        exhibited by the simulations reported in Fig.\ \ref{fig:sim_grid_da}. 
        The ringlet's dynamical timescale $\tau_{dyn}$ is time the when its nonlinearity parameter $q$ exceeds $0.6$.
        Sample ringlets have initial $2.5\times10^{-5}\le\Delta a\le2.0\times10^{-4}$,
        $10^{-14}\le m_r\le10^{-10}$, and $10^{-14}\le \nu_s\le10^{-11}$.
        Horizontal black lines indicated the observed eccentricity gradients 
        exhibited by the Maxwell, Titan, Laplace, and Huygens ringlets at Saturn 
        (\citealt{Netal14}, \citealt{Fetal16}, \citealt{SH16}). Dotted line is the $e'=\sqrt{3}/2$ threshold.
    }
\end{figure}


\subsubsection{variations with ringlet viscosity}
\label{subsec:viscosity-variations}

Figure \ref{fig:wt_prime_vs_time} shows the periapse twist $\tilde{\omega}' \simeq ea\Delta\tilde{\omega}/\Delta a$ 
versus time for six ringlets
having the same initial $e_0$, $\Delta a$, $m_r$ and $\nu_b$ as the nominal ringlet but with differing shear
viscosities $\nu_s$, and that plot shows that twist $|\tilde{\omega}'|$ varies with $\nu_s$. Which indicates that if 
the twist $|\tilde{\omega}'|$ could be observed in a self-confining ringlet, then 
the ringlet's viscosity could then be inferred. 

\begin{figure}
    \plotone{figs/wt_prime_vs_time.pdf}
    \caption{
        \label{fig:wt_prime_vs_time}
        Periapse twist $\tilde{\omega}'$ is plotted versus time $t/\tau_\nu$ for six ringlets
        having the same initial $e_0$, $\Delta a$, $m_r$ as the nominal ringlet
        but differing shear viscosities $\nu_s$ that range over $0.5\nu_n\le\nu_s\le5\nu_n$
        where $\nu_n=1.0\times10^{-13}$ is the nominal ringlet's shear visocity.
        The time-evolution of these ringlets' other orbit elements,  $\Delta a$, $e$, $e'$, and $q$,
        are very similar to that exhibited by the nominal ringlet, 
        Figs.\ \ref{fig:da_nominal}--\ref{fig:de_prime_nominal}. 
    }
\end{figure}


\subsubsection{variations with initial eccentricity}
\label{subsec:e-variations}

Additional simulations illustrate how outcomes depend upon the ringlet's initial eccentricity $e_0$.
Figure \ref{fig:e0} shows seven simulations of the nominal ringlet that all have identical
physical properties (mass $m_r$, viscosity $\nu_s$, and initial width $\Delta a$) but differing initial $e_0$ ranging
over $0\le e_0\le 0.04$. That plot shows that higher-$e$ ringlets settle into the
self-confining state sooner than the lower-$e$ ringlets. This is because the higher-$e$ ringlet's
secular gravitational perturbation of itself drives its eccentricity gradient and hence $q$ towards 
$q\simeq\sqrt{3}/2$ faster than the lower-$e$ ringlets. Consequently, 
higher-$e$ ringlets tend to be narrower than lower-$e$ ringlets because they will have had
less time to spread before settling into self-confinement. Also note that the $e_0=0$ ringlet
(blue curve) experiences zero secular gravitational perturbations, so its $q$ is always zero
and is destined to spread forever.

\begin{figure}
    \plotone{figs/e0.pdf}
    \caption{
        \label{fig:e0}
        Simulations of seven nominal ringlets having a variety of initial eccentricities $0\le e_0\le 0.04$.
        Curves give each ringlet's semimajor axis width $\Delta a$ in units of its initial $\Delta a_0$
        versus time $t/\tau_\nu$, and show that higher-$e$ ringlets settle into self-confinement
        sooner than the lower-$e$ ringlets.
    }
\end{figure}

The next Subsection will examine the rate at which
a self-confining ringlet's eccentricity $e$ decays over time due to viscosity, 
and all ringlets shown in Figs.\ \ref{fig:a_nominal}--\ref{fig:e_prime_q_vs_time}
have eccentricities $e$ that decay at the expected rates.


\subsection{eccentricity damping}
\label{subsec:eccentricity}

Viscous friction within the ringlet is a result of dissipative collisions among ringlet particles.
Particle collisions generate heat that is radiated into space, and the source of that radiated energy
is the ringlet's orbital energy $E_r=-m_rGM/2a + E_{sg}$ where $m_r$ is the ringlet's total
mass, $a$ its semimajor axis, and $E_{sg}$ is the ringlet's energy due to its self gravity
which is contant when the ringlet is self-confining. 
Collisions conserve angular momentum, so the ringlet's total angular momentum
$L_r=m_r\sqrt{GMa(1-e^2)}$ is constant so $dL_r/dt=0$ implies that
\begin{equation}
    \label{eqn:e2-dot}
    \frac{de^2}{dt} \simeq \frac{1}{a}\frac{da}{dt}
\end{equation}
to lowest order in the ringlet's small eccentricity $e$.
The ringlet's energy dissipation rate is $\dot{E}_r = dE_r/dt=m_rGM\dot{a}/2a^2$ so
$\dot{a}\simeq2\dot{E_r}/m_ra\Omega^2$ and 
\begin{equation}
    \label{eqn:de2/dt}
    \frac{de^2}{dt} \simeq \frac{2\dot{E_r}}{m_r a^2\Omega^2}
\end{equation}
where $GM\simeq a^3\Omega^2$ to lowest order in $J_2$. Also note that
the surface area of energy dissipation within a viscous disk is
\begin{equation}
    \delta = -\nu_s\sigma(r\omega')^2
\end{equation}
(\citealt{P81}) where $\omega=v_\theta/r$ is the angular velocity and 
$\omega'=\partial\omega/\partial r$ its radial gradient.

Now consider a small tangential segment within the ringlet whose length is $d\ell=rd\varphi$ where 
$\varphi$ is the segment's longitude measured from the ringlet's periapse and
$d\varphi$ is the small segment's angular extent. The segment's area is 
$dA=\Delta rd\ell=r\Delta r d\varphi$
where $\Delta r$ is the ringlet's radial width.
The rate at which that patch's viscosity dissipates orbital energy is $d\dot{E_r}=\delta dA$, so
the ringlet's total energy dissipation rate is 
$\dot{E}_r = \oint d\dot{E_r}$ when integrated about the ringlet's circumference, and so
$\dot{E}_r = -2\nu_s\lambda\int_0^\pi r^3\omega'^2d\varphi$
since the ringlet's linear density $\lambda=\sigma\Delta r\simeq m_r/2\pi a$. So the
total energy loss rate due to ringlet viscosity becomes
\begin{equation}
    \label{eqn:dE_r/dt}
    \dot{E}_r \simeq -\frac{9}{4}I(e')m_r\nu_s\Omega^2
\end{equation}
when Eqn.\ (\ref{eqn:domega-dr}) is used to replace $\omega'$, and the integral 
\begin{equation}
    I(e') =  \frac{1}{\pi}\int_0^\pi\left(\frac{1-\frac{4}{3}e'\cos\varphi}{1-e'\cos\varphi}\right)^2d\varphi .
\end{equation}
Note that  $I(e')$ is of order
unity except when $e'$ is very close to 1, and numerical evaluation shows that $I(e')\simeq0.889$
when $e'=\sqrt{3}/2$.

Inserting Eqn.\ (\ref{eqn:dE_r/dt}) into (\ref{eqn:de2/dt}) then yields the rate at which $e^2$
is damped,
\begin{equation}
    \label{eqn:de2/dt_v2}
    \frac{de^2}{dt} = -\frac{9I\nu_s}{2a^2}
\end{equation}
which is easily integrated to obtain
\begin{equation}
    \label{eqn:e(t)}
    e(t) = e_0\sqrt{1-\frac{t}{\tau_e}}
\end{equation}
where $e_0$ is the ringlet's initial eccentricity and
\begin{equation}
    \label{eqn:tau_e}
    \tau_e = \frac{2a^2e_0^2}{9I\nu_s}
\end{equation}
is the ringlet's eccentricity damping timescale.
These expectations are also confirmed in Fig.\ \ref{fig:e_vs_tau_e}, which plots
$e(t)/e_0$ versus time $t/\tau_e$ for the sample of 28 simulations 
shown in Fig.\ \ref{fig:e_prime_q_vs_time}, with the dashed curve
indicating the theoretial predictions of Eqns.\ (\ref{eqn:e(t)}--\ref{eqn:tau_e}).
That all simulated curves have slopes similar to the dashed line tells us that
Eqns.\ (\ref{eqn:e(t)}--\ref{eqn:tau_e}) are a good indicator of outcomes across
a wide variety of ringlet parameters.
\begin{figure}
    \plotone{figs/e_vs_tau_e.pdf}
    \caption{
        \label{fig:e_vs_tau_e}
        Plot of $e(t)/e_0$ versus time $t/\tau_e$ for the sample of 28
        simulations shown in Fig.\ \ref{fig:e_prime_q_vs_time},
        while the dashed curve is the expected behavior, Eqn.\ (\ref{eqn:e(t)}).
        This sample of simulations have $\tau_e$ ranging over 
        $2.5\times10^6 \le \tau_e \le 2.5\times10^9$.
    }
\end{figure}

So viscosity circularizes the ringlet in time $\tau_e$, during which time the ringlet's
semimajor axis will have shrunk by $\Delta a=\dot{a}\tau_e=-e_0^2a$ by Eqns (\ref{eqn:e2-dot})
and (\ref{eqn:de2/dt_v2}), so the ringlet's
fractional drift inwards due to viscous damping is
\begin{equation}
    \label{eqn:delta-a}
    \frac{\Delta a}{a} = -e_0^2,
\end{equation}
which is small. And after the ringlet's inner edge damps to zero, its eccentricity gradient $e'$
will then shrink over time, angular momentum flux reversal will diminish, and the ringlet's 
viscous spreading will resume. So self-confinement of narrow eccentric ringlets is only temporary
after all, until time $\tau_e$ has elapsed, which is $\tau_e/2\pi\sim1.6\times10^6$ orbits for 
the nominal model considered here, which is only $\sim10^3$ years for a ringlet orbiting at 
$a\sim10^{10}$ cm about Saturn. Recall from Section \ref{subsec:nominal}
that the viscous lifetime of a non-self-confining
nominal ringlet is only $\tau_\nu/2\pi\sim500$ orbits, so self-confinement evidently
extends the lifetime of a narrow eccentric ringlet by an additional 
factor of $\sim3000$. But self-confinement
does not solve the ringlet's lifetime problem, because self-confinement is ultimately
defeated by viscous damping of the ringlet's eccentricity.


\subsection{number of streamlines $N_s$}
\label{subsec:num_streamlines}

When the simulated ringlet is composed of $N_s=2$ streamlines, the ringlet's
evolution is largely analytic ({\it c.f.}\ \citealt{BGT82, BGT83}), and the analytic
predictions provide excellent benchmark tests for the epi\_int\_lite integrator. This subsection 
assesses whether the results obtained for the simpler $N_s=2$ ringlet also apply to
more realistic ringlets having $N_s>2$.

Figures \ref{fig:e_vs_da_streamlines}--\ref{fig:delta_wt_vs_da_streamlines} recomputes the nominal
ringlet's evolution for ringlets having a range of streamlines, $2\le N_s\le14$, all of which
have $q$-evolution very similar to that exhibited by the $N_s=2$ simulation seen in 
Fig.\ \ref{fig:de_prime_nominal}.  Figure \ref{fig:e_vs_da_streamlines} 
plots each streamline's eccentricity $e$
versus their relative semimajor axis $\Delta a=a-\bar{a}$, which shows that
all simulated ringlet's have the same eccentricity gradient $e'$ regardless
of the number of streamlines $N_s$.
Ditto for the ringlets' relative longitudes of periapse, $\Delta\tilde{\omega}$ when
plotted versus $\Delta a$, Fig.\ \ref{fig:delta_wt_vs_da_streamlines},
which shows that all simulated ringlets have comparable gradients in $\tilde{\omega}$.
The only noteworthy difference seen here is that the smaller $N_s\le 3$ simulations
do not resolve the extra peripase twist that is seen at the edges of the higher resolution simulations.
Except for this one distinction,
the evolution of the $N_s>2$ ringlets are very similar to that exhibited by nominal
ringlet composed of $N_s=2$ streamlines.

\begin{figure}
    \plotone{figs/e_vs_da_streamlines.pdf}
    \caption{
        \label{fig:e_vs_da_streamlines}
        A suite of ringlets are evolved, these ringlets have the same properties as the
        nominal ringlet except they contain varied numbers of streamlines $2\le N_s\le14$,
        with $N_s$ indicated by the plot legend.
        Simulated ringlets' final eccentricities $e$ are plotted versus their relative
        semimajor axis $\Delta a=a-\bar{a}$ where $\bar{a}$ is the ringlet's mean semimajor axis. 
    }
\end{figure}
\begin{figure}
    \plotone{figs/delta_wt_vs_da_streamlines.pdf}
    \caption{
        \label{fig:delta_wt_vs_da_streamlines}
        Simulated ringlets' final relative longitudes of periapse 
        $\Delta\tilde{\omega} = \tilde{\omega} - \bar{\omega}$ are plotted versus their relative
        semimajor axis $\Delta a=a-\bar{a}$, where $\bar{a}$ is the ringlet's mean semimajor axes 
        and $\bar{\omega}$ is their mean longitude of periapse, and number of streamlines $N_s$
        is indicated in the plot legend. 
    }
\end{figure}

\subsubsection{surface densities and sharp edges}
\label{subsubsec:sharp_edges}

The main shortcoming of the $N_s=2$ simulation is that it reveals nothing about a ringlet's
possible sharp edges, since a two-streamline ringlet always has artificially sharp edges.
To examine this further, Fig.\ \ref{fig:periapse_sigma_vs_r_streamlines.pdf} shows the radial 
surface density profile along the periapse direction for the $2\le N_s\le14$ ringlets 
that were simulated in Section \ref{subsec:num_streamlines}. That plot shows that all ringlets'
edges are sharp after they arrive in the self-confining state, regardless of $N_s$. When self-confining,
each streamline is approximately equidistant (within $25\%$) from their neighbors, which causes
ringlet surface density $\sigma$ to be remain constant (within $25\%$) in the ringlet interior, 
which then plummets to zero beyond the ringlet's boundaries. Ditto for the surface density
profiles seen along the ringlets' apoapse direction, Fig.\ \ref{fig:apoapse_sigma_vs_r_streamlines.pdf}.
In summary, the self-confining ringlet examined here has a smooth radial surface density profile that is
concave--down, with surface density maxima at the ringlet' center, and sharp edges.

Note that if a vicous ringlet were instead unconfined, then its positive angular momentum flux
would have repelled the edgemost streamlines away from the interior streamlines, 
which in turn would have caused the ringlet's concave--down surface density profile
to taper smoothly to zero at its edges, as is seen in Fig.\ 1 of \cite{P81}.
Meanwhile, the other ringlet models, those that rely on unseen shepherd satellites to maintain confinement, all
exhibit smooth surface densities having a variety of concavities, and all have sharp edges (\citealt{GT79, CG00, ME02}).

In comparison, the Maxwell ringlet's radial optical depth profiles can be approximately 
described as concave--downish with a spiral density wave riding on top (\citealt{Netal14}). The Titan
ringlet is opaque so its concavity is unknown, but it does have sharp edges (\citealt{Netal14}).
The Huygens ringlet, which has a tiny eccentricity gradient, is concave--up with sharp edges 
(\citealt{Fetal16}), while the Laplace ringlet is possibly concave--down with sharp edges
and lots of internal structure. Main point: all well-observed narrow eccentric planetary ringlets
have a radial optical depth structure that is much more complicated than is exhibited by
published models to date.

\begin{figure}
    \plotone{figs/periapse_sigma_vs_r_streamlines.pdf}
    \caption{
        \label{fig:periapse_sigma_vs_r_streamlines.pdf}
        Radial surface density profiles $\sigma(\Delta r)$ 
        of the $2\le N_s\le14$ ringlets of Section \ref{subsec:num_streamlines}
        plotted along the ringlets' direction of periapse.
        Surface density $\sigma$ is shown in units of the ringlet's mean surface density $\bar{\sigma}$,
        and radial distance $\Delta r = r - r_{\text{mid}}$ is measured relative to chord's midpoint
        and in unit of the ringlet's semimajor axis width $\Delta a=a_{\text{outer}}-a_{\text{inner}}$. 
    }
\end{figure}
\begin{figure}
    \plotone{figs/apoapse_sigma_vs_r_streamlines.pdf}
    \caption{
        \label{fig:apoapse_sigma_vs_r_streamlines.pdf}
        Radial surface density profiles $\sigma(\Delta r)$ 
        of the $2\le N_s\le14$ simulations, plotted along the ringlets' direction of apoapse.
        Note that the ringlets' apoapse surface densities are about 13 times smaller
        than their periapse $\sigma$, Fig.\ \ref{fig:periapse_sigma_vs_r_streamlines.pdf},
        because they are also wider by that amount.
    }
\end{figure}

\section{ringlet origin scenarios}
\label{sec:origin}

This Section uses the preceeding results to speculate about the origin of narrow eccentric planetary ringlets.
If these ringlets are truly self-confining, then they are extremely young,
only $\sim{\cal O}(1000)$ years old (Section \ref{subsec:eccentricity}). Our preferred 
least-speculative origin scenario argues that
the ringlet precursor was originally an especially large ring particle that was orbiting elsewhere such as
within a dense planetary ring, presumably as an embedded moonlet (reference?). If that embedded moonlet happened to 
form near the edge of the dense planetary ring, then the moonlet's reaction to
the shepherding torque that it exerts on the ring [which is the radial integral of Eqn.\ (79) of \cite{GT82}] would
cause that moonlet to migrate towards and then beyond the ring's edge. That small moonlet would also be
susceptible to collisional disruption by an interplanetary impactor (reference?), and if so then
differential orbital motion would have quickly sheared the resulting collisional debris into a narrow planetary ringlet.
Although the parent moonlet presumably was in a circular orbit due to the planetary tide, the interplanetary
impactor was on a hyperbolic orbit, so the just-formed planetary ringlet's angular momentum vector
would be the vector sum of the parent moonlet's angular momentum and the hyperbolic impactor's. 
Point is, angular momentum conservation implies that the just-formed planetary ringlet's orbit
would likely have a small eccentricity and inclination (is that $inc$ a problem here?). Viscous particle-particle
collisions would then cause that young ringlet to spread radially (Fig.\ \ref{fig:da_nominal}) 
while its self-gravity pumped up its eccentricity gradient (Fig. \ref{fig:de_prime_nominal}), 
which if sufficiently vigorous would cause it to settle into the self-confining state, 
which is sharp-edged like most observed narrow eccentric ringlets. Nonetheless, self-confinement is temporary
and lasts only $\sim{\cal O}(1000)$ years due to ringlet viscosity damping the ringlet's eccentricity until its inner edge gets circularized and the ringlet's $q$ to drop below the self-confinement threshold, $q\simeq\sqrt{3}/2$.

So in summary, this purportedly least-speculative ringlet origin scenario
implies that dense planetary rings are frequently forming embedded moonlets that then migrate, 
due to its gravitational shepherding of that ring, towards and then beyond the nearby ring-edge
where they are later collisionally disrupted by an interplanetary impactor. The resulting debris quickly
shears out into a low-eccentricity ringlet whose self-gravity excites its eccentricity gradient
until it settles temporarily into the sharp-edged self-confining state. The next step in any assessment of the viability
of this so-called least-speculative ringlet origin scenario would require estimating various timescales,
namely the moonlet's formation timescale, the moonlet's migration timescale, its lifetime versus collisional
distruption, and the ringlet's self-confinement lifetime, to confirm that none of these lifetimes are so
long as to cast doubt on this ringlet origin scenario. That more detailed analysis is deferred to a followup
study. 










\section{summary of findings}
\label{sec:summary}

Main findings:

\begin{enumerate}

\item Simulations show that narrow eccentric ringlets having a wide variety of initial 
physical properties (mass, initial width, viscosity) do evolve
into the self-confining state (see Fig.\ \ref{fig:sim_grid_da}), 
provided that the ringlet's initial eccentricity is sufficiently high (Section \ref{subsec:e-variations}). 
Self-gravity causes the ringlet's eccentricity gradient $e'$ to grow over time until they near the
$\simeq e'\simeq\sqrt{3}/2\simeq0.866$ threshold where viscous angular momentum flux
reversal is nearly complete and there is {\em almost} no radial orbit-averaged
transmission of angular momentum due to the ringlet's viscous friction.

Simulations also show that these self-confining ringlets all have a small 
periapse twist $|w'|\ll e'$ so that the
ringlet's nonlinearity parameter is dominated by their eccentricity gradient
$q=\sqrt{e'^2 + \tilde{\omega}'^2}\simeq e'$.

\item Self confining ringlets have $L_E=0$

\item partial self confinement

\end{enumerate}


\section{followup studies}
\label{sec:followup}

Possible followup studies...

\acknowledgments
\section{Acknowledgments}
\label{sec:Acknowledgments}

This research was supported by the National Science Foundation via Grant No.\ AST-1313013.


\appendix

\section{Appendix A}
\label{sec:Appendix A}

The drift step implemented in the original epi\_int code utilized the epicyclic equations of \cite{BL94}
to advance an unperturbed particle along its trajectory about an oblate planet. Those equations
convert a particle's spatial coordinates $r, \theta, v_r, v_\theta$
into epicyclic orbit elements $a,e,M,\tilde{\omega}$ that are easily advanced by timestep
$\Delta t$, after which the epicyclic elements are converted back to spatial coordinates
with an accuracy to ${\cal O}(e^2)$. Note though that those epicyclic equations are not reversible;
every conversion from spatial to epicyclic coordinates or vice-versa also introduces an 
error of order ${\cal O}(e^3)$, and the accumulation of those errors causes the orbits of the epi\_int particles to 
slowly drift over time.  But that slow numerical drift
was not problematic for the relatively short epi\_int simulations reported in \cite{HS13} that evolved
Saturn's B ring for typically $\sim10^4$ orbit periods. However the viscous self-gravitating
ringlets considered here must be evolved for $10^5$ to $10^6$ orbit periods in order for self-confinement to occur,
and simulations having those longer execution times were being defeated by this orbital drift. 
The remedy is to derive an alternate set of epicyclic equations that
{\em are} reversible such that the epi\_int\_lite's drift step will not be a significant
source of numerical error.

Begin with the equation of motion for an unperturbed particle in orbit about the central planet, 
\begin{equation}
    \label{eqn:EOM}
    \ddot{\mathbf{r}}=-\nabla\Phi
\end{equation}
where the planet's gravitational potential is
\begin{equation}
    \label{eqn:Phi}
    \Phi = -\frac{GM}{r} - \frac{J_2GMR^2}{2r^3}
\end{equation}
where $M$ is the planet's mass, $R$ its radius, $J_2$ its second zonal harmonic due to oblateness,
and with higher-order $J_{n>2}$ terms ignored here. The angular part of Eqn.\ (\ref{eqn:EOM}) is
\begin{equation}
    \label{eqn:angular_EOM}
    \frac{1}{r}\frac{d}{dt}(r^2\dot{\theta}) = -\frac{\partial\Phi}{\partial\theta} = 0
\end{equation}
so the particle's specific angular momentum $h=r^2\dot{\theta}=r v_\theta$ is conserved {\it i.e.}
\begin{equation}
    \label{eqn:theta-dot}
    \dot{\theta} = \frac{h}{r^2},
\end{equation}
and inserting that into the radial part of Eqn.\ (\ref{eqn:EOM}) yields
\begin{equation}
    \label{eqn:radial_EOM}
    \ddot{r} = \frac{h^2}{r^3} -\frac{GM}{r^2} - \frac{3J_2GMR^2}{2r^4}.
\end{equation}
Solving the above to first-order accuracy ${\cal O}(e^1)$ ordinarily yields
\begin{eqnarray}
    \label{eqn:linearized_soln}
    r &\simeq& a(1-e\cos M) \nonumber\\
    \theta &\simeq& \frac{\Omega}{\kappa}(M+2e\sin M) + \tilde{\omega} \nonumber\\
    v_r = \dot{r} &\simeq& ea\kappa\sin M \nonumber\\
    v_\theta = r\dot{\theta} &\simeq& a\Omega(1 + e\cos M)
\end{eqnarray}
where
\begin{equation}
    \label{eqn:omega}
    \Omega(a) = \sqrt{\frac{GM}{a^3}}\left[1 + \frac{3}{2}J_2\left(\frac{R}{a}\right)^2\right]^{1/2}
\end{equation}
is the particle's mean orbital frequency and 
\begin{equation}
    \label{eqn:kappa}
    \kappa(a) = \sqrt{\frac{GM}{a^3}}\left[1 - \frac{3}{2}J_2\left(\frac{R}{a}\right)^2\right]^{1/2}
\end{equation}
its epicyclic frequency. In the above,  $a$, $e$, $\tilde{\omega}$ and $M=\kappa t$ are the epicyclic orbit elements
that have errors of order ${\cal O}(e^2)$. Equations (\ref{eqn:linearized_soln})
are easily inverted, which would then provide epicyclic
orbit elements $a$, $e$, $\tilde{\omega}$, $M$ as functions of the particle's 
spatial coordinates $r$, $\theta$, $v_r$, $v_\theta$, but applying them
here would be even more problematic since the conversion from spatial coordinates to
orbit elements would accrue ${\cal O}(e^2)$ errors during every drift step, which is even
worse that the rate at which epi\_int's drift step accrues errors.

The remedy is to choose an alternate set of equations (\ref{eqn:linearized_soln}) that also
solve the equation of motion (\ref{eqn:EOM}) to the same accuracy while  
satisfying Eqn.\ (\ref{eqn:theta-dot}) exactly. Begin with the above expression for the
angular coordinate, $\theta=(\Omega/\kappa)(M+2e\sin M) + \tilde{\omega}$,
so that $\dot{\theta} =\Omega(1+2e\cos M)$. Then require the constant $h=r^2\dot{\theta}=a^2\Omega$
be satisfied exactly, which provides the revised expression for $r$ as a function of orbit elements,
$r = a/\sqrt{1+2e\cos M}$, noting that this expression differs from that in Eqn.\ (\ref{eqn:linearized_soln}) 
by an amount of order ${\cal O}(e^2)$ which is this solution's allowed error.
It is then straightforward to complete the revisions to Eqn.\ (\ref{eqn:linearized_soln}):
\begin{eqnarray}
    \label{eqn:revised_linearized_soln}
    r &=& \frac{a}{\sqrt{1+2e\cos M}} \nonumber\\
    \theta &=& \frac{\Omega}{\kappa}(M+2e\sin M) + \tilde{\omega} \nonumber\\
    v_r = \dot{r} &=& \frac{ea\kappa\sin M}{(1 + 2e\cos M)^{3/2}}  = ea\kappa\sin M\left(\frac{r}{a}\right)^3\nonumber\\
    v_\theta = r\dot{\theta} &=& a\Omega\sqrt{1 + 2e\cos M} = \frac{a^2\Omega}{r}
\end{eqnarray}
where $\Omega$ and $\kappa$ are Eqns.\ (\ref{eqn:omega}--\ref{eqn:kappa}) and $M=\kappa t$.

Eqns.\ (\ref{eqn:revised_linearized_soln}) must also be inverted so
that the particle's epicyclic orbit elements $a, e, \tilde{\omega}, M$ can be obtained
from the particle's spatial coordinates $r, \theta, v_r, v_\theta$, which is done via
the following:

\begin{enumerate}

\item Calculate the particle's specific angular momentum $h=r v_\theta$ and then solve 
$h^2 = a^4\Omega^2 = GMa[1 + (3J_2/2)(R/a)^2]$
for the particle's semimajor axis $a$, which is
\begin{equation}
    \label{eqn:a}
    a = R\left(C + \sqrt{C^2 - 3J_2/2}\right)
\end{equation}
where constant $C=h^2/2GMR$.

\item Calculate $\Omega$ and $\kappa$ using Eqns.\ (\ref{eqn:omega}--\ref{eqn:kappa}).

\item Note that $e\sin M = (v_r/a\kappa)(a/r)^3$ and $e\cos M = \onehalf[(a/r)^2 - 1]$, which 
provides the particle's eccentricity via $e = \sqrt{(e\sin M)^2 + (e\cos M)^2}$
and its mean anomaly $M$ inferred from $\tan M = e\sin M/e\cos M$.

\item The particle's longitude of periapse is 
$\tilde{\omega} = \theta - (\Omega/\kappa)(M + 2e\sin M)$.

\end{enumerate}
These equations satisfy the equation of motion (\ref{eqn:EOM}) to accuracy ${\cal O}(e)$ and
with errors of order ${\cal O}(e^2)$. To advance a particle during epi\_int\_lite's drift step from 
time $t$ to time $t + \Delta t$, use the above steps 1-4 to convert the particle's spatial
coordinates $r, \theta, v_r, v_\theta$ to epicyclic orbit elements $a,e,M,\tilde{\omega}$,
update the particle's mean anomaly via $M\rightarrow\kappa(t + \Delta t)$, and then use
Eqns.\ (\ref{eqn:revised_linearized_soln}) to compute the particle's revised spatial coordinates.
And because steps 1-4 convert coordinates to elements exactly, epi\_int\_lite's drift step 
is not a significant source of numerical error.

\section{Appendix B}
\label{sec:Appendix B}

Derive the angular shear $\partial\omega/\partial r$.
The angular velocity $\omega=v_\theta/r=\Omega(a/r)^2$ by Eqn.\ (\ref{eqn:revised_linearized_soln}) so
$\omega \simeq\Omega(1+2e\sin\varphi)$
to first order in $e$ and relative longitude $\varphi=\theta -\tilde{\omega}\simeq M$.
Thus $\partial\omega/\partial a \simeq -(3\Omega/2a)(1-\frac{4}{3}e'\cos\varphi)$
since $J_2$ and $e$ are small though
the eccentricity gradient $e'=a(\partial e/\partial a)$ might not be small. Likewise
$r\simeq a(1-e\cos\varphi)$ so $\partial r/\partial a\simeq 1-e'\cos\varphi$ and so
\begin{equation}
    \label{eqn:domega-dr}
    \omega' = \frac{\partial\omega}{\partial r} = 
        \frac{\partial\omega}{\partial a}/\frac{\partial r}{\partial a}
        \simeq -\left(\frac{3\Omega}{2a}\right) \frac{1-\frac{4}{3}e'\cos \varphi}{1-e'\cos \varphi},
\end{equation}
which interestingly changes sign near periapse when the eccentricity gradient is large enough, $e'>3/4$.

\section{Appendix C}
\label{sec:Appendix C}

Compare epi\_int\_lite to theoretical predictions

\section{Appendix D}
\label{sec:Appendix D}

This examines the viscous evolution of a narrow eccentric non-gravitating
ringlet that is identical to the nominal ringlet of Section \ref{subsec:nominal} but
with ringlet self-gravity neglected and $J_2=0$.
As the orange curve in Fig.\ \ref{fig:da_nogravity} shows, the non-gravitating ringlet's
radial width $\Delta a$ grows over time due to ringlet viscosity, 
long after the nominal self-gravitating ringlet (blue curve)
has settled into the self-confining state by time $t\sim40\tau_\nu$. This is due to the
ringlet's secular gravitational perturbations of itself,
which tends to excites the ringlet's outer streamline's eccentricity at the expense
of the inner streamline (see Fig.\ \ref{fig:e_nominal}) until the ringlet eccentricity gradient $e'$
(blue curve in Fig.\ \ref{fig:de_prime_nogravity}) grows beyond the
limit required for complete angular momentum flux reversal 
that results in the ringlet's radial confinement (dotted line). 
Note that viscosity also excites the non-gravitating
ringlet's eccentricity gradient some (orange curve), but not sufficiently to halt the ringlet's 
viscous spreading.

\begin{figure}
    \plotone{figs/da_nogravity.pdf}
    \caption{
        \label{fig:da_nogravity}
        Blue curve is the nominal ringlet's semimajor axis width $\Delta a$ versus time $t$,
        and this ringlet's radial spreading ceases by time $t\sim40\tau_\nu$ when it's self-gravity
        has excited the ringlet's eccentricity gradient $e'$ sufficiently; 
        see blue curve in Fig.\ \ref{fig:de_prime_nogravity}. Orange curve shows that
        the non-gravitating ringlet's $\Delta a$ grows without limit due to the ringlet's
        much lower eccentricity gradient. Note that planetary oblateness would
        cause the non-gravitating streamlines to precess differentially and eventually cross
        when $J_2>0$, so the non-gravitating simulation also sets $J_2=0$ 
        to avoid differential precession.
    }
\end{figure}

\begin{figure}
    \plotone{figs/de_prime_nogravity.pdf}
    \caption{
        \label{fig:de_prime_nogravity}
        Eccentricity gradient $e'$ versus time $t$ for self-gravitating nominal ringlet (blue curve),
        and non-gravitating ringlet (orange).
    }
\end{figure}


\section{Appendix F}
\label{sec:Appendix F}

Viscous and gravitational energy transport...


\bibliography{jmh_bibliography}
\end{document}


\end{document}




\section{Trash}


\subsection{improved drift step around oblate planet}
\label{subsec:drift}


\subsection{kick}
\label{subsec:kick}



Epi\_int is a drift-kick integrator, and such integrators alternate between drifting
(ie advancing) a particle along its unperturbed trajectory, with each drift followed
by a velocity kick that accounts for all other perturbing forces such as ring self gravity
and ring viscosity. Drifting a particle efficiently along its unperturbed trajectory around
an oblate planet requires an analytic
expression for that trajectory, and epi\_int utilized the \cite{BL94} solution
that requires, at every timestep, the conversion of the particle's spatial coordinates
and velocities into geometric orbit elements, with the drifted particle's
orbit elements then converted back to spatial coordinates every timestep. That conversion is accurate
to order ${\cal O}(e^2)$ where $e$ is the particle's geometric eccentricity, but the
conversion from spatial coordinates to orbit elements and back is not reversible, which
means that the drifted particle's trajectory acquires an ${\cal O}(e^3)$ error
every timestep. Although the accumulation of this error was too slow to significantly impact
epi\_int's B ring simulations spanning $10^4$ orbit periods, this error does preclude 
using that code to simulate the much slower viscous evolution of ringlets over $10^6$ orbit periods.

To avoid this accumulation of drift errors, Section \ref{sec:drift-kick} derives an alternate set of geometric
orbit elements that describe the particle's unperturbed motion around an oblate planet. Note though
the conversion of spatial coordinates to the new geometric orbit elements is exact
and reversible, and so epi\_int\_lite's drift step is not a significant source of error. 

The chief principal guiding the developement of epi\_int\_lite is that the code be
accurate to solve the problem at hand while also being as simple
as possible so that the code can be developed, tested, and executed as swiftly as possble.
With this in mind, several simplifying approximations are made and are detailed
below in Section \ref{subsec:approximations} and they simplify code development and
shorten run times significantly. Section \ref{sec:testing} then assesses the impact
of those approximations, and shows that they are truly negligable and do not affect outcomes
or conclusions.

Although the epi\_int integrator was well-suited for evolving the
B ring edge over the $\sim3\times10^4$ orbits needed to monitor the B ring's response to Mimas
perturbations, that code lacked sufficient numerical accuracy to evolve a ringlet
during the $\sim3\times10^6$? orbits needed to track its slow radial spreading due to ringlet viscosity.
That inability was traced to epi\_int's drift step, and Section ? describes
how that is mitigated in the new code.
