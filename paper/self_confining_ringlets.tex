%self_confining_ringlets.tex
%
%by Joe Hahn et al, jhahn@spacescience.org, 20 November 2018.
%
%draft ApJ paper on epi_int_lite simulations of self confining ringlets.
%
%to compile:
%    pdflatex self_confining_ringlets
%    bibtex self_confining_ringlets
%    pdflatex self_confining_ringlets


\documentclass[preprint]{aastex62}

%bibliography, see
%http://ads.harvard.edu/pubs/bibtex/astronat/doc/html/astronat_3.html
\usepackage{natbib}
%\citestyle{aa}

%journal
\received{not yet}
\revised{not yet}
\accepted{not yet}
\submitjournal{Somewhere, eventually}

%short names
\shorttitle{Narrow Eccentric Ringlets}
\shortauthors{Hahn et al.}


\begin{document}


%title page
\title{N-body simulations of the Self--Confinement of 
Viscous Self--Gravitating Narrow Eccentric Planetary Ringlets}

\correspondingauthor{Joseph M. Hahn}
\email{jhahn@spacescience.org}

\author{Joseph M. Hahn}
\affiliation{Space Science Institute}

\author{Douglas P. Hamilton}
\affiliation{University of Maryland}

\author{Thomas Rimlinger}
\affiliation{University of Maryland}

\author{Lucy Luu}
\affiliation{University of Maryland}


%abstract
\begin{abstract}

The following shows how narrow eccentric planetary ringlets can evolve into a 
self--confining state.

\end{abstract}


%keywords
\keywords{editorials, notices --- miscellaneous --- catalogs --- surveys --- update, me}


\section{Introduction}
\label{sec:intro}

Narrow eccentric planetary ringlets have properties both interesting and
not well understood: sharp edges,
sizable eccentricity gradients, and a confinement mechanism that
inhibits radial spreading due to ring viscosity.
Prevailing ringlet confinement mechanisms include: 
unseen shepherd satellites (reference), periapse pinch (ref), self gravity (ref), and
self-confinement (ref). This study uses N-body simulations to show how narow
self--gravitating ringlets can evolve into a self-confining state.

\section{Ringlet confinement mechanisms}
\label{sec:confinement}

This section explains the pros and cons of the various ringlet confinement mechanisms,
and then motivates the possibility that ringlets are self confining. That possibility
is explored further via numerical simulations using the epi\_int\_lite N-body integrator.

\section{epi\_int\_lite}
\label{sec:epi_int_lite}

Epi\_int\_lite is a child of the epi\_int N-body integrator that was used to
simulate the outer edge of Saturn's B ring that is sculpted by satellite perturbations
\citep{HS13}. The new code is very similar to its parent but differs in three significant ways:
({\it i}.) epi\_int\_lite is written in python and recoded for more efficient execution,
({\it ii}.) epi\_int\_lite uses a more accurate drift step for 
unperturbed motion around an oblate planet (detailed in Appendix \ref{sec:Appendix A}),
and ({\it iii}.) epi\_int\_lite uses the $C=1$ approximation that is justified below 
(Appendix \ref{sec:Appendix B}).

Otherwise epi\_int\_lite's treatment of ring self--gravity and viscosity are identical
to that used by the parent code; see \cite{HS13} for additional details. The epi\_int\_lite 
source code is available at https://github.com/joehahn/epi\_int\_lite, and the
code's numerical quality is assessed in Appendix \ref{sec:Appendix C}
where the output of several numerical experiments are compared against theoretical predictions.

Calculations performed by epi\_int\_lite use natural units with gravitation constant $G=1$, 
central primary mass $M=1$, and the ringlet's inner edge has initial radius
$r_0=1$, and so the ringlet masses $m_r$ and radii $r$ quoted below are in units of $M$ and $r_0$.
Converting code output from natural units to physical units requires choosing	
physical values for $M$ and $r_0$ and multiplying accordingly, and when this text does so
it assumes the primary's mass is Saturn's,  $M=5.68\times10^{29}$ gm, and a typical
ring radius of $r_0=1.0\times10^{10}$ cm.


\section{N-body simulations of viscous gravitating ringlets}
\label{sec:nbody}

This Section describes a suite of N-body simulations of narrow viscous gravitating planetary
ringlets, to highlight the range of initial ringlet conditions the do evolve into
a self-confining state, and those that do not.

\subsection{nominal model}
\label{subsec:nominal}

This nominal model evolves into the self-confining state...

\acknowledgments

acknowledgments...

\appendix

\section{Appendix A}
\label{sec:Appendix A}

Derive the more accurate drift step used by epi\_int\_lite...

\section{Appendix B}
\label{sec:Appendix B}

Detail the $C=1$ approximation used by epi\_int\_lite, and show that the errors
associated with this approximation are negligable...

\section{Appendix C}
\label{sec:Appendix C}

Compare epi\_int\_lite to theoretical predictions

\subsection{radial spreading of viscous viscous}
\label{subsec:spreading}

Show that ringlet viscosity causes circular non-gravitating ringlet to
spread at the expected rate...

%bibliography
%\bibliographystyle{apj}
\bibliography{jmh_bibliography}
\end{document}


\end{document}




\section{Trash}


\subsection{improved drift step around oblate planet}
\label{subsec:drift}


\subsection{kick}
\label{subsec:kick}



Epi\_int is a drift-kick integrator, and such integrators alternate between drifting
(ie advancing) a particle along its unperturbed trajectory, with each drift followed
by a velocity kick that accounts for all other perturbing forces such as ring self gravity
and ring viscosity. Drifting a particle efficiently along its unperturbed trajectory around
an oblate planet requires an analytic
expression for that trajectory, and epi\_int utilized the \cite{BL94} solution
that requires, at every timestep, the conversion of the particle's spatial coordinates
and velocities into geometric orbit elements, with the drifted particle's
orbit elements then converted back to spatial coordinates every timestep. That conversion is accurate
to order ${\cal O}(e^2)$ where $e$ is the particle's geometric eccentricity, but the
conversion from spatial coordinates to orbit elements and back is not reversible, which
means that the drifted particle's trajectory acquires an ${\cal O}(e^3)$ error
every timestep. Although the accumulation of this error was too slow to significantly impact
epi\_int's B ring simulations spanning $10^4$ orbit periods, this error does preclude 
using that code to simulate the much slower viscous evolution of ringlets over $10^6$ orbit periods.

To avoid this accumulation of drift errors, Section \ref{sec:drift-kick} derives an alternate set of geometric
orbit elements that describe the particle's unperturbed motion around an oblate planet. Note though
the conversion of spatial coordinates to the new geometric orbit elements is exact
and reversible, and so epi\_int\_lite's drift step is not a significant source of error. 

The chief principal guiding the developement of epi\_int\_lite is that the code be
accurate to solve the problem at hand while also being as simple
as possible so that the code can be developed, tested, and executed as swiftly as possble.
With this in mind, several simplifying approximations are made and are detailed
below in Section \ref{subsec:approximations} and they simplify code development and
shorten run times significantly. Section \ref{sec:testing} then assesses the impact
of those approximations, and shows that they are truly negligable and do not affect outcomes
or conclusions.




Although the epi\_int integrator was well-suited for evolving the
B ring edge over the $\sim3\times10^4$ orbits needed to monitor the B ring's response to Mimas
perturbations, that code lacked sufficient numerical accuracy to evolve a ringlet
during the $\sim3\times10^6$? orbits needed to track its slow radial spreading due to ringlet viscosity.
That inability was traced to epi\_int's drift step, and Section ? describes
how that is mitigated in the new code.
