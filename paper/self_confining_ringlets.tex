%self_confining_ringlets.tex
%
%by Joe Hahn et al, jhahn@spacescience.org, 20 November 2018.
%
%draft ApJ paper on epi_int_lite simulations of self confining ringlets.
%
%to compile:
%    pdflatex self_confining_ringlets
%    bibtex self_confining_ringlets
%    pdflatex self_confining_ringlets


\documentclass[preprint]{aastex62}

%bibliography, see
%http://ads.harvard.edu/pubs/bibtex/astronat/doc/html/astronat_3.html
\usepackage{natbib}
%\citestyle{aa}

%journal
\received{not yet}
\revised{not yet}
\accepted{not yet}
\submitjournal{Somewhere, eventually}

%short names
\shorttitle{Narrow Eccentric Ringlets}
\shortauthors{Hahn et al.}


\begin{document}


%title page
\title{Self--Confinement of Narrow Eccentric Ringlets}

\correspondingauthor{Joseph M. Hahn}
\email{jhahn@spacescience.org}

\author{Joseph M. Hahn}
\affiliation{Space Science Institute}

\author{Douglas P. Hamilton}
\affiliation{University of Maryland}

\author{Thomas Rimlinger}
\affiliation{University of Maryland}

\author{Lucy Luu}
\affiliation{University of Maryland}


%abstract
\begin{abstract}

Abstractity abstract stract...

\end{abstract}


%keywords
\keywords{editorials, notices --- miscellaneous --- catalogs --- surveys --- update, me}


\section{Introduction}
\label{sec:intro}

Narrow eccentric ringlets have properties that are both interesting and
not well understood: they have very sharp edges,
they have sizable eccentricity gradients, and the mechanism that
prevents their radial spreading due to ring viscosity is of debate.
Prevailing ringlet confinement mechanisms include: 
unseen shepherd satellites (reference), periapse pinch (ref), self gravity (ref), and
self-confinement (ref). The following uses N-body simulations to investigate whether narrow
eccentric ringlets might be self-confining.

\section{Ringlet confinement mechanisms}
\label{sec:confinement}

This section will explain the pros and cons of the various ringlet confinement mechanisms,
and then motivate the possibility that ringlets are self confining. That possibility
is then investigated via numerical simulations that use the epi\_int\_lite N-body integrator.

\section{epi\_int\_lite}
\label{sec:epi_int_lite}

epi\_int\_lite is a child of the epi\_int N-body integrator that was used to
simulate the outer edge of Saturn's B ring as it is sculpted by perturbations
from satellite Mimas \cite{HS13}. Although the epi\_int integrator is well-suited to evolve the
B ring edge over the $10^4$ orbits needed to monitor the B ring's response to Mimas
perturbations, that code was unable evolve with sufficient accuracy over the $10^6$ orbits needed
to investigate a viscous ringlet's slow radial spreading, and that inability was
traced to epi\_int's drift step. 

Epi\_int is a drift-kick integrator, and such integrators alternate between drifting
(ie advancing) a particle along its unperturbed trajectory, with each drift followed
by a velocity kick that accounts for all other perturbing forces such as ring self gravity
and ring viscosity. Drifting a particle efficiently along its unperturbed trajectory around
an oblate planet requires an analytic
expression for that trajectory, and epi\_int utilized the \cite{BL94} solution
that requires, at every timestep, the conversion of the particle's spatial coordinates
and velocities into geometric orbit elements, with the drifted particle's
orbit elements then converted back to spatial coordinates every timestep. That conversion is accurate
to order ${\cal O}(e^2)$ where $e$ is the particle's geometric eccentricity, but the
conversion from spatial coordinates to orbit elements and back is not reversible, which
means that the drifted particle's trajectory acquires an ${\cal O}(e^3)$ error
every timestep. Although the accumulation of this error was too slow to significantly impact
epi\_int's B ring simulations spanning $10^4$ orbit periods, this error does preclude 
using that code to simulate the much slower viscous evolution of ringlets over $10^6$ orbit periods.

To avoid this accumulation of drift errors, Section \ref{sec:drift-kick} derives an alternate set of geometric
orbit elements that describe the particle's unperturbed motion around an oblate planet. Note though
the conversion of spatial coordinates to the new geometric orbit elements is exact
and reversible, and so epi\_int\_lite's drift step is not a significant source of error. 

\subsection{code design philosophy}
\label{sec:philosophy}

The chief principal guiding the developement of epi\_int\_lite is that the code be
accurate to solve the problem at hand while also being as simple
as possible so that the code can be developed, tested, and executed as swiftly as possble.
With this in mind, several simplifying approximations are made and are detailed
below in Section \ref{subsec:approximations} and they simplify code development and
shorten run times significantly. Section \ref{sec:testing} then assesses the impact
of those approximations, and shows that they are truly negligable and do not affect outcomes
or conclusions.

\subsection{drift-kick}
\label{sec:drift-kick}


\subsection{approximations}
\label{subsec:approximations}

\section{testing the approximations}
\label{sec:testing}

\acknowledgments

acknowledgments...

\appendix

\section{Appendix}

appendix...


%bibliography
%\bibliographystyle{apj}
\bibliography{jmh_bibliography}
\end{document}


\end{document}
