%self_confining_ringlets.tex
%
%by Joe Hahn et al, jhahn@spacescience.org, 20 November 2018.
%
%draft ApJ paper on epi_int_lite simulations of self confining ringlets.
%
%to compile:
%    pdflatex self_confining_ringlets
%    bibtex self_confining_ringlets
%    pdflatex self_confining_ringlets


\documentclass[preprint]{aastex62}

%bibliography, see
%http://ads.harvard.edu/pubs/bibtex/astronat/doc/html/astronat_3.html
\usepackage{natbib}
%\citestyle{aa}

%journal
\received{not yet}
\revised{not yet}
\accepted{not yet}
\submitjournal{Somewhere, eventually}

%short names
\shorttitle{Narrow Eccentric Ringlets}
\shortauthors{Hahn et al.}


\begin{document}


%title page
\title{N-body simulations of the Self--Confinement of 
Viscous Self--Gravitating Narrow Eccentric Planetary Ringlets}

\correspondingauthor{Joseph M. Hahn}
\email{jhahn@spacescience.org}

\author{Joseph M. Hahn}
\affiliation{Space Science Institute}

\author{Douglas P. Hamilton}
\affiliation{University of Maryland}

\author{Thomas Rimlinger}
\affiliation{University of Maryland}

\author{Lucy Luu}
\affiliation{University of Maryland}


%abstract
\begin{abstract}

N-body simulations are used
to illustrate how narrow eccentric planetary ringlets can evolve into a 
self--confining state.

\end{abstract}


%keywords
\keywords{editorials, notices --- miscellaneous --- catalogs --- surveys --- update, me}


\section{Introduction}
\label{sec:intro}

Narrow eccentric planetary ringlets have properties both interesting and
not well understood: sharp edges,
sizable eccentricity gradients, and a confinement mechanism that
opposes radial spreading due to ring viscosity.
Prevailing ringlet confinement mechanisms include: 
unseen shepherd satellites (reference), periapse pinch (ref), self gravity (ref), and
self-confinement (ref). This study uses N-body simulations to show how a narrow
self--gravitating ringlet can evolve into a self-confining state.

\section{Ringlet confinement mechanisms}
\label{sec:confinement}

This section will explain the pros and cons of the various ringlet confinement mechanisms,
and then motivates the possibility that ringlets are self confining. That possibility
is explored further via numerical simulations using the epi\_int\_lite N-body integrator.

\section{epi\_int\_lite}
\label{sec:epi_int_lite}

Epi\_int\_lite is a child of the epi\_int N-body integrator that was used to
simulate the outer edge of Saturn's B ring as it is sculpted by satellite perturbations
\citep{HS13}. The new code is very similar to its parent but differs in three significant ways:
({\it i}.) epi\_int\_lite is written in python and recoded for more efficient execution,
({\it ii}.) epi\_int\_lite uses a more accurate drift step for 
unperturbed motion around an oblate planet (detailed in Appendix \ref{sec:Appendix A}),
and ({\it iii}.) epi\_int\_lite uses the $C=1$ approximation that is justified below 
(Appendix \ref{sec:Appendix B}).

Otherwise epi\_int\_lite's treatment of ring self--gravity and viscosity are identical
to that used by the parent code; see \cite{HS13} for additional details. The epi\_int\_lite 
source code is available at https://github.com/joehahn/epi\_int\_lite, and the
code's numerical quality is assessed in Appendix \ref{sec:Appendix C}
where the output of several numerical experiments are compared against theoretical expectations.

Calculations performed by epi\_int\_lite use natural units with gravitation constant $G=1$, 
central primary mass $M=1$, and the ringlet's inner edge has initial radius
$r_0=1$, and so the ringlet masses $m_r$ and radii $r$ quoted below are in units of $M$ and $r_0$.
Converting code output from natural units to physical units requires choosing	
physical values for $M$ and $r_0$ and multiplying accordingly, and when this text does so
it assumes Saturn's mass $M=5.68\times10^{29}$ gm and a characteristic
ring radius $r_0=1.0\times10^{10}$ cm. Simulation time $t$ is in units of $T_{\text{orb}}/2\pi$
where $T_{\text{orb}} = 2\pi\sqrt{r_0^3/GM}$ is the orbit period at $r_0$, 
so divide $t$ by $2\pi$ and then multiply
by $T_{\text{orb}}$ to convert simulation time from natural to physical units.
The simulated particles' motions during the drift step are also
sensitive to the $J_2$ portion of the primary's non-spherical gravity component 
(see Appendix \ref{sec:Appendix B}), and all simulations
adopt a Saturn-like value of $J_2=0.01$ and $R_p=r_0/2$ where $R_p$ is the planet's
mean radius.

Initially all particles are assigned to various streamlines across the simulated ringlet. A streamline
is a closed eccentric path around the primary, and the $N_p$ particles in a given
streamline are initially assigned a common semimajor axis $a$ and eccentricity $e$, 
with uniform spacing in longitude. Most of the simulations described below
employ only $N_s=2$ streamlines, so that the model output can be benchmarked against
theoretical treatments that also treat the ringlet as two gravitating rings
(e.g.\ \citealt{BGT83}). But the following also performs a few higher-resolution simulations
using $N_s=11$ streamlines, to demonstrate that the $N_s=2$ treatment is perfectly
adequate and reproduces all the relevant dynamics. All simulations use $N_p=241$ particles 
per streamline, and the total number of particles is $N_sN_p$.
Note that the assignment of particles to a given streamline is merely
for labeling purposes, as particles are still free to wander in response
to the ring’s internal forces, namely, ring gravity and viscosity. But as \cite{HS13} as well
as this work shows, the simulated ring stays coherent and highly organized throughout the 
simulation such that particles on the same streamline do not pass each other longitudinally,
nor do they cross adjacent streamlines. Because the simulated ringlet stays highly organized,
there is no radial or longitudinal mixing of the ring particles, and simulated particles preserve
their streamline membership over time.


\section{N-body simulations of viscous gravitating ringlets}
\label{sec:nbody}

This Section describes a suite of N-body simulations of narrow viscous gravitating planetary
ringlets, to highlight the range of initial ringlet conditions the do evolve into
a self-confining state, and those that do not.

\subsection{nominal model}
\label{subsec:nominal}

Figure \ref{fig:a_nominal} shows the semimajor axis evolution of what is referred to
as the nominal model since this ringlet readily evolves into a self-confining state.
The simulated ringlet is composed of $N_s=2$ streamlines having $N_p=241$ particles
per streamline, and the integrator timestep is $\Delta t=0.5$ in natural units, so
the integrator samples the particles' orbits $2\pi/\Delta t\simeq13$ times per orbit, and this
ringlet is evolved for $6.7\times10^5$ orbits, which requires 40 minutes execution time
on a 5 year old laptop. The ringlet's mass is
$m_r=2\times10^{-9}$, its shear viscosity is $\nu_s=1\times10^{-12}$, and its
bulk viscosity is $\nu_b=1.5\nu_s$. The ringet's initial radial width is
$\Delta a_0 = 5\times10^{-4}$, its initial eccentricity is $e=0.03$, and its
eccentricity gradient is initially zero. A convenient measure of time is the ringlet's
viscous radial spreading timescale
\begin{equation}
\label{eqn:viscous-timesscale}
    \tau_\nu=\frac{\Delta a_0^2}{12\nu_s}, 
\end{equation}
which can be inferred from Eqn.\ (2.13) of \cite{P81}. 
This simulation's viscous timescale is $\tau_\nu=2.1\times10^4$ in natural units
or $\tau_\nu/2\pi=3.3\times10^3$ orbital periods. If this ringlet were orbiting Saturn
at $r_0=1.0\times10^{10}$ cm then the simulated ringlet's physical mass
would be $m_r=1.1\times10^{21}$ gm which is equivalent to the mass of a $64$ km radius iceball assuming
a volume density $\rho=1$ gm/cm$^3$, and the ringlet's initial radial width would be
$\Delta a_0 = 5\times10^{-4}r_0=50$ km. This ringlet's
orbit period would be $T_{orb}=2\pi\sqrt{r_0^3/GM}=9.0$ hours in physical units, so 
the ringlet's viscous timescale is $\tau_\nu=3.4$ years 
which indicates that shear viscosity is $\nu_s=\Delta a_0^2/12\tau_\nu = 1.9\times10^4$ cm$^2$/sec
when evaluated in physical units. 
This ringlet's initial surface density would be $\sigma=m_r/2\pi r_0\Delta a_0=3500$ gm/cm$^2$, but
Figs.\ \ref{fig:a_nominal}--\ref{fig:da_nominal} show that shrinks by a factor of 3 as the 
ringlet's sememajor axis width $\Delta a$ grows via viscous spreading until it settles into
the anticipated self-confining state at time $t\sim15\tau_\nu$.
So the so-called nominal ringlet is probably overdense and overly viscous compared to known 
planetary ringlets,
but that is by design so that the simulated ringlet quickly settles into the self-confining state.
Section XX also shows how outcomes scale when a wide variety of alternate initial masses, orbits,
and viscosities are also considered.

\begin{figure}
\plotone{figs/a_nominal.pdf}
\caption{Evolution of the nominal ringlet's semimajor axes $a$
versus time $t$ in units of the ringlet's viscous time $\tau_\nu$.
This ringlet is composed of $N_s=2$ streamlines,
and the outer (blue) and inner (green) streamlines' semimajor axes are plotted relative
to their mean $a_{\text{mean}}$, and displayed in units of the ringlet's
initial width $\Delta a_0 = 5\times10^{-4}$ in natural units ({\it i.e.}\ $G=M=r_0=1$).
The simulated ringlet has total mass $m_r=2\times10^{-9}$, shear viscosity $\nu_s=1\times10^{-12}$,
and initial eccentricity $e=0.03$. See Section \ref{subsec:nominal} to convert
$m_r$, $a$ and $\nu_s$ from natural units to physical units.
\label{fig:a_nominal}}
\end{figure}

\begin{figure}
\plotone{figs/da_nominal.pdf}
\caption{
\label{fig:da_nominal}
The nominal ringlet's semimajor axis width $\Delta a = a_{\text{outer}} - a_{\text{inner}}$ over time, 
in units of its initial radial width $\Delta a_0$.}
\end{figure}

Figure \ref{fig:e_nominal} shows that the outer streamline's eccentricity grows at the
expense of the inner streamline's, and this is a consequence the self-gravitating ringlet's
secular perturbations of itself, which is also demonstrated in Appendix \ref{sec:Appendix D}. 
Figure \ref{fig:de_nominal} shows
the ringlet's eccentricity difference $\Delta e = e_{\text{outer}} - e_{\text{inner}}$
and longitude of periapse difference
$\Delta\tilde{\omega} = \tilde{\omega}_{\text{outer}} - \tilde{\omega}_{\text{inner}}$,
which both settle into equilibrium values when the ringlet arrives in the self-confining
state. It is convenient to recast these orbit element differences as dimensionless gradients
\begin{equation}
    \label{eqn:e_prime}
    e' = a\frac{de}{da}
    \qquad\mbox{and}\qquad
    \tilde{\omega}' = ea\frac{d\tilde{\omega}}{da}
\end{equation}
as both terms contribute to the nonlinearity parameter of \cite{BGT83}:
\begin{equation}
    \label{eqn:q}
    q = \sqrt{e'^2 + \tilde{\omega}'^2}.
\end{equation}
See also Fig.\ \ref{fig:de_prime_nominal} which plot's the nominal
ringlet's dimensionless eccentricity gradient $e'$, dimensionless periapse twist $\tilde{\omega}'$
and nonlinearity parameter $q$ versus time. All simulations examined here have
$|\tilde{\omega}'|\ll|e'|$ so that $q\simeq|e'|$, and all simulated self-confining ringlets
have a positive eccenticity gradient and a negative periapse twist such that
the outer ringlet's periapse trails the inner ringlet's, consistent with the findings of
\cite{BGT83}.\vfil

\begin{figure}
\plotone{figs/e_nominal.pdf}
\caption{
\label{fig:e_nominal}
The nominal ringlet's eccentricity evolution.}
\end{figure}

\begin{figure}
    \plotone{figs/de_nominal.pdf}
    \caption{
        \label{fig:de_nominal}
        The nominal ringlet's eccentricity difference $\Delta e = e_{\text{outer}} - e_{\text{inner}}$
        and longitude of periapse difference
        $\Delta\tilde{\omega} = \tilde{\omega}_{\text{outer}} - \tilde{\omega}_{\text{inner}}$.
    }
\end{figure}

\begin{figure}
    \plotone{figs/de_prime_nominal.pdf}
    \caption{
        \label{fig:de_prime_nominal}
        The nominal ringlet's dimensionless eccentricity gradient $e' = a\Delta e/\Delta a$
        (dashed orange curve), dimensionless periapse twist $\tilde{\omega}' = ea\Delta\tilde{\omega}/\Delta a$
        (blue curve), and nonlinearity parameter $q=\sqrt{e'^2 + \tilde{\omega}'^2}$
        (green curve) versus time $t/\tau_\nu$. Dotted red line 
        is the threshold for self-confinement in a non-gravitating ringlet, $e'=\sqrt{3}/2\simeq0.866$
    }
\end{figure}\vfil


\subsection{viscous angular momentum transport}
\label{subsec:viscous_flux}

The nominal ringlet's evolution is more easily understood when the ringlet's viscous flux of angular momentum is 
considered; that flux is 
\begin{equation}
    \label{eqn:F_nu_theta}
    F_{\nu}(r, \theta) = -\nu_s \sigma r^2\frac{\partial\omega}{\partial r}
\end{equation}
\citep{HS13} when written as a function of spatial coordinates and
angular velocity $\omega=\dot{\theta}$ (Eqn.\ XX). If we imagine a small arc of 
ring material of transverse length $d\ell$, 
then $F_{\nu}d\ell$ would be the torque that arc exert exerts on
ring matter just exterior, due to viscous friction.
Which is also the rate that friction transmits angular momentum radially across that arc.
And when $F_{\nu}$ is evaluated along a single eccentric streamline of semimajor axis $a$, 
the above simplifies to
\begin{equation}
    \label{eqn:F_nu_varphi}
    F_{\nu}(a, \varphi) = F_{\nu,0}\frac{1-\frac{4}{3}e'\cos\varphi}{(1-e'\cos\varphi)^2}
\end{equation}
(see \citealt{BGT82} and Appendix \ref{sec:Appendix E})
where $F_{\nu,0}=\frac{3}{2}\nu_s\sigma_0a\Omega$ is the viscous angular momentum flux through a
circular streamline of semimajor axis $a$ and angular speed $\Omega(a)$ and assuming
$|\tilde{\omega}'|\ll e'$ so that $q\simeq e'$. Integrating the above
around the streamline's circumference then yields its angular momentumum luminosity,
\begin{equation}
    \label{eqn:L_nu}
    L_{\nu}(a) = \oint F_\nu(e', \varphi)rd\varphi = L_{\nu,0}\frac{1-\frac{4}{3}e'^2}{(1-e'^2)^{3/2}},
\end{equation}
which is the torque that one streamline exerts on its exterior neighbor due to viscous friction
(\citealt{BGT82} and Appendix \ref{sec:Appendix E}),
where $L_{\nu,0}=3\pi\nu_s\sigma_0a^2\Omega$ viscous angular momentum luminosity of a circular streamline.

\citet{BGT82} examine angular momentum transport through a viscous eccentric but non-gravitating ringlet, 
and use Eqns.\ (\ref{eqn:F_nu_varphi}--\ref{eqn:L_nu}) to show that this transport has three 
regimes distinguished by the ringlet's $e'$:

\begin{enumerate}

\item $e'<3/4$, for which the ringlet's viscous angular momentum flux $F_\nu(\theta)>0$
at all ringlet longitudes $\theta$. The ringlet's viscous
angular momentum luminosity $L_\nu>0$, so viscous friction transports angular momentum radially outwards,
so inner ring matter evolves to smaller orbits while exterior ring matter evolves outwards, and
the ringlet spreads radially.

\item $3/4\le e'<\sqrt{3}/4$. In this regime there is a range of longitudes $\theta$
where the viscous angular momentum flux is reversed such that $F_\nu(\theta)<0$. 
Nonetheless $L_\nu$, which is the orbit-average of $F_\nu(\theta)$,
is positive and the ringlet still spreads radially, albeit slower than when $e'<0.75$.

\item $e'\ge\sqrt{3}/4$. Viscous angular momentum flux reversal is complete such that $L_\nu<0$,
viscous friction transports angular momentum radially inwards, and the ringlet
shrinks radially. But if $e'=\sqrt{3}/4\simeq0.866$ then $L_\nu=0$ and the ringlet's
radial evolution ceases, and the viscous but non-gravitating ringlet is self confining.

\end{enumerate}
 
Note though that the nominal ringlet's eccentricity gradient
exceeds the $e'=\sqrt{3}/4\simeq0.866$ threshold (which is
the dotted red line in Fig.\ \ref{fig:de_prime_nominal}) when it settles into
self-confinement. This is due to the ringlet's self-gravity,
which also transports angular momentum radially through the ringlet.

Figure \ref{fig:F_nu_nominal} shows the nominal ringlet's viscous angular momentum flux
$F_\nu$ versus relative longitude $\varphi=\theta-\tilde{\omega}$ at selected times $t$.
Early in the ringlet's evolution when time $t \le 4\tau_\nu$ (blue, orange, green, and red curves),
the ringlet is in regime 1 since $e'<0.75$ and $F_\nu(\theta)>0$ at all longitudes.
But by time $t = 6\tau_\nu$ (dark purple curve), this ringlet's eccentricity gradient exceeds $0.75$,
and angular momentum flux reversal $F_\nu(\theta)<0$ occurs near periapse where $|\varphi|\simeq0$
where the ringlet is most overdense due to its eccentricity gradient, see 
alsoFig.\ \ref{fig:radial_sigma_nominal};
this ringlet is in regime 2 and its radial spreading begins to deaccelerate. 
And by time $t = 10\tau_\nu$ (brown curve), this ringlet is seemingly in regime 3
since $e'>0.866$, so one might expect the ringlet's spreading to have stalled by
now, but keep in mind that the above analysis ignores any transport
of angular momentum via ringlet self--gravity. Figure (\ref{fig:da_nominal})
shows that this gravitating ringlet's spreading ceases soon after time $t \simeq20\tau_\nu$, 
at which point $e'=0.9$ (light purple curve), angular momentum flux reversal is complete,
and the ringlet's total angular momentum luminosity $L=0$
(see Figs.\ \ref{fig:angular_momentum_luminosity_nominal}--\ref{fig:angular_momentum_luminosity_zoom_nominal}).

An interesting question is why a gravitating ringlet's viscous angular momentum luminosity is
positive event though $e'>0.866$. Dunno...

\begin{figure}
    \plotone{figs/F_nu_nominal.pdf}
    \caption{
        \label{fig:F_nu_nominal}
        Nominal ringlet's viscous angular momentum flux $F_{\nu}(\varphi)$,
        Eqn.\ (\ref{eqn:F_nu_varphi}),  is plotted versus ringlet
        relative longitude $\varphi=\theta-\tilde{\omega}$ along inner (double-check this)
        streamline, with the ringlet's eccentricy gradient $e'$ also indicated
        at selected times $t/\tau_\nu$. Also explain how $F_{\nu}(\varphi)$ is computed.
    }
\end{figure}

\begin{figure}
    \plotone{figs/radial_sigma_nominal.pdf}
    \caption{
        \label{fig:radial_sigma_nominal}
        Radial profiles of the nominal ringlet's surface density $\sigma(\varphi)$ at time $t/\tau_\nu=50$
        when the ringlet is self-confining. Each surface density profile is plotted versus radial distance $r$ 
        relative to $r_{mid}$ which is the ringlet's midpoint along relative longitude $\varphi = \theta-\tilde{\omega}=0$,
        with these radial distances $r - r_{mid}$ also measured in units of the ringlet's final semimajor axis width $\Delta a$.
        Radial surface density profiles are plotted along the ringlet's periapse ($\varphi=0$, blue curve), which is 
        where the ringlet's streamlines are most concentrated and surface denisity $\sigma$ is
        greatest due to the ringlet's eccentricity gradient $e'$, at quadrature ($|\varphi|=\pi/2$, green curve),
        and at apoapse ($|\varphi|=\pi$, orange curve) where streamlines have their greatest separation
        and ringlet surface density is lowest. This ringlet's surface density contrast, between periapse and
        apoapse, is 14.
    }
\end{figure}

\begin{figure}
    \plotone{figs/angular_momentum_luminosity_nominal.pdf}
    \caption{
        \label{fig:angular_momentum_luminosity_nominal}
        Nominal ringlet's viscous angular momentum luminosity $L_\nu/L_{\nu,0}$ (blue curve) versus time $t/\tau_\nu$,
        where $L_{\nu,0}$ is circular ring's viscous angular momentum luminosity, 
        as well as the ringlet gravitational angular momentum luminosity $L_g$ (orange curve) in units of $L_{\nu,0}$.
    }
\end{figure}

\begin{figure}
    \plotone{figs/angular_momentum_luminosity_zoom_nominal.pdf}
    \caption{
        \label{fig:angular_momentum_luminosity_zoom_nominal}
        Figure \ref{fig:angular_momentum_luminosity_nominal} is replotted to highlight that the ringlet's 
        viscous angular momentum luminosity $L_\nu$ (blue curve)
        always stays positive (indicating that the viscous transport of angular momentum is radially outwards) while balancing
        the ringlet's negative ({\it i.e.}\ inwards) gravitational angular momentum luminosity 
        $L_g$ (orange) such that the ringlet's total angular momentum luminosity $L=L_\nu+L_g=0$  (green curve)
        after time $t\gg20\tau_\nu$.
    }
\end{figure}

\begin{figure}
    \plotone{figs/F_vs_longitude_nominal.pdf}
    \caption{
        \label{fig:F_vs_longitude_nominal}
        The nominal ringlet's viscous angular momentum flux $F_{\nu}(\varphi)$ (blue curve), plotted in units of $F_{\nu,0}$
        and versus relative longitude $\varphi$ at time $t=50\tau_\nu$, as well as the ringlet's gravitational
        angular momentum flux $F_{g}(\varphi)$ (orange curve).
    }
\end{figure}

Figure \ref{fig:F_vs_longitude_nominal} shows how viscosity and self-gravity transport angular
momentum across a self-confining ringlet at various longitudes $\varphi$. That figure shows that ringlet viscosity
transports angular momentum inwards {\it i.e.}\ $F_{\nu}(\varphi)<0$ near periapse, and outwards 
with $F_{\nu}(\varphi)>0$ at all other longitudes. Which is rather distinct from the ringlet's gravitational
transport, which has $F_{g}(\varphi)<0$ as
ring-matter travels towards periapse, and outwards $F_{g}(\varphi)>0$ after periapse.
Despite these spatial differences, the influence of both forces still sum to zero
in the orbit-integrated sense {\it i.e.}\ $\oint (F_\nu+F_g)rd\varphi=0$ after the ringlet has
settled into the self-confining state.

Ring viscosity and self-gravity can also transport energy across the ring, 
and that is assessed in Appendix \ref{sec:Appendix F}.

\acknowledgments

This research was supported by the National Science Foundation via Grant No.\ AST-1313013.

\appendix

\section{Appendix A}
\label{sec:Appendix A}

Derive the more accurate drift step used by epi\_int\_lite...

\section{Appendix B}
\label{sec:Appendix B}

Detail the $C=1$ approximation used by epi\_int\_lite, and show that the errors
associated with this approximation are negligible...

\section{Appendix C}
\label{sec:Appendix C}

Compare epi\_int\_lite to theoretical predictions

\section{Appendix D}
\label{sec:Appendix D}

This examines the viscous evolution of a narrow eccentric non-gravitating
ringlet that is identical to the nominal ringlet of Section \ref{subsec:nominal} but
with ringlet self-gravity neglected and $J_2=0$.
As the orange curve in Fig.\ \ref{fig:da_nogravity} shows, the non-gravitating ringlet's
radial width $\Delta a$ grows steadily over time due to ringlet viscosity, 
long after the nominal self-gravitating ringlet (blue curve)
has settled into the self-confining state by time $t\sim15\tau_\nu$. This is due to the
ringlet's secular gravitational perturbations of itself,
which tends to excites the ringlet's outer streamline's eccentricity at the expense
of the inner streamline (see Fig.\ \ref{fig:e_nominal}) until the ringlet eccentricity gradient $e'$
(blue curve in Fig.\ \ref{fig:de_prime_nogravity}) grows beyond the
limit required for complete angular momentum flux reversal 
that results in the ringlet's radial confinement (dotted line). 
Note that viscosity also excites the non-gravitating
ringlet's eccentricity gradient some (orange curve), but insufficient to halt the ringlet's 
viscous spreading.

\begin{figure}
    \plotone{figs/da_nogravity.pdf}
    \caption{
        \label{fig:da_nogravity}
        Blue curve is the nominal ringlet's semimajor axis width $\Delta a$ versus time $t$,
        and this ringlet's radial spreading ceases by time $t\sim15\tau_\nu$ when it's self-gravity
        has excited the ringlet's eccentricity gradient $e'$ sufficiently; 
        see blue curve in Fig.\ \ref{fig:de_prime_nogravity}. Orange curve shows that
        the non-gravitating ringlet's $\Delta a$ grows without limit due to the ringlet's
        much lower eccentricity gradient. Note that planetary oblateness would
        cause the non-gravitating streamlines to precess differentially and eventually cross
        when $J_s>0$, so the non-gravitating simulation also sets $J_2=0$ 
        to avoid differential precession.
    }
\end{figure}

\begin{figure}
    \plotone{figs/de_prime_nogravity.pdf}
    \caption{
        \label{fig:de_prime_nogravity}
        blah
    }
\end{figure}

\section{Appendix E}
\label{sec:Appendix E}

This Appendix will use the orbit elements derived in Appendix \ref{sec:Appendix A} to
derive Eqn.\ \ref{eqn:F_nu_varphi} from \ref{eqn:F_nu_theta}, and then Eqn.\ (\ref{eqn:L_nu}).

\section{Appendix F}
\label{sec:Appendix F}

Viscous and gravitational energy transport...


\bibliography{jmh_bibliography}
\end{document}


\end{document}




\section{Trash}


\subsection{improved drift step around oblate planet}
\label{subsec:drift}


\subsection{kick}
\label{subsec:kick}



Epi\_int is a drift-kick integrator, and such integrators alternate between drifting
(ie advancing) a particle along its unperturbed trajectory, with each drift followed
by a velocity kick that accounts for all other perturbing forces such as ring self gravity
and ring viscosity. Drifting a particle efficiently along its unperturbed trajectory around
an oblate planet requires an analytic
expression for that trajectory, and epi\_int utilized the \cite{BL94} solution
that requires, at every timestep, the conversion of the particle's spatial coordinates
and velocities into geometric orbit elements, with the drifted particle's
orbit elements then converted back to spatial coordinates every timestep. That conversion is accurate
to order ${\cal O}(e^2)$ where $e$ is the particle's geometric eccentricity, but the
conversion from spatial coordinates to orbit elements and back is not reversible, which
means that the drifted particle's trajectory acquires an ${\cal O}(e^3)$ error
every timestep. Although the accumulation of this error was too slow to significantly impact
epi\_int's B ring simulations spanning $10^4$ orbit periods, this error does preclude 
using that code to simulate the much slower viscous evolution of ringlets over $10^6$ orbit periods.

To avoid this accumulation of drift errors, Section \ref{sec:drift-kick} derives an alternate set of geometric
orbit elements that describe the particle's unperturbed motion around an oblate planet. Note though
the conversion of spatial coordinates to the new geometric orbit elements is exact
and reversible, and so epi\_int\_lite's drift step is not a significant source of error. 

The chief principal guiding the developement of epi\_int\_lite is that the code be
accurate to solve the problem at hand while also being as simple
as possible so that the code can be developed, tested, and executed as swiftly as possble.
With this in mind, several simplifying approximations are made and are detailed
below in Section \ref{subsec:approximations} and they simplify code development and
shorten run times significantly. Section \ref{sec:testing} then assesses the impact
of those approximations, and shows that they are truly negligable and do not affect outcomes
or conclusions.




Although the epi\_int integrator was well-suited for evolving the
B ring edge over the $\sim3\times10^4$ orbits needed to monitor the B ring's response to Mimas
perturbations, that code lacked sufficient numerical accuracy to evolve a ringlet
during the $\sim3\times10^6$? orbits needed to track its slow radial spreading due to ringlet viscosity.
That inability was traced to epi\_int's drift step, and Section ? describes
how that is mitigated in the new code.
